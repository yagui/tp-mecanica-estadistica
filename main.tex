\documentclass[a4paper]{article}

\usepackage[spanish]{babel}
\usepackage[utf8]{inputenc}
\usepackage[centertags]{amsmath}
\usepackage{amssymb}
\usepackage{amsfonts}
\usepackage{amsthm}
\usepackage{graphicx}
\usepackage[colorinlistoftodos]{todonotes}
\usepackage{geometry}
\usepackage{dsfont} % para la funcion indicador \mathds{1}
\usepackage{verbatim}

\geometry{
 left=25mm,
 right=25mm,
 top=20mm,
 bottom=30mm,
 }


\newcommand{\expo}[1]{\exp{\left\lbrace #1 \right\rbrace}}
\newcommand{\indi}[1]{\mathds{1}\left\lbrace #1 \right\rbrace}
\newcommand{\umbral}{\theta}
\newcommand{\uom}{\underline{\omega}}
\newcommand{\uOm}{\underline{\Omega}}
\newcommand{\Pif}{\Pi\!\left(\frac{\theta-C_i(\uom)}{\sigma_i(\uom)}\right)}
\newcommand{\reals}{\mathds{R}}
\newcommand{\enteros}{\mathds{Z}}
\newcommand{\naturales}{\mathds{N}}
\newcommand{\abs}[1]{\left|#1\right|}
\newcommand{\norm}[1]{\left\lVert#1\right\rVert}
\newcommand{\var}[1]{\text{var}\left[#1\right]}
\newcommand{\cov}[1]{\text{cov}\left[#1\right]}
\newcommand{\prob}[1]{\mathbb{P}\left(#1\right)}
\newcommand{\textoCircle}[1]{\raisebox{.5pt}{\textcircled{\raisebox{-.9pt}{#1}}}}
\newcommand{\esperanza}[1]{\mathbb{E} \left[#1\right]}

\theoremstyle{definition}
\newtheorem{prop}{Proposición}
\newtheorem{thm}{Teorema}
\newtheorem{propiedad}{Propiedad}

\title{Modelo de red neuronal a tiempo discreto con trenes de spikes. Dinámica con ruido}

\author{Guillermo Delmas, Ariel Burman}

\date{\today}

\begin{document}
\maketitle

% Test de fuentes
% 
% \begin{align*}
% RQSZ \\
% \mathcal{RQSZL} \\
% \mathfrak{RQSZL} \\
% % \mathbb{PRQSZL}
% \mathds{PRQSZL}
% \end{align*}

% \begin{align*}
% 3x^2 \in R \subset Q \\
% % \mathnormal{3x^2 \in R \subset Q} \\
% \mathrm{3x^2 \in R \subset Q} \\
% % \mathit{3x^2 \in R \subset Q} \\
% % % \mathbf{3x^2 \in R \subset Q} \\
% \mathsf{3x^2 \in R \subset Q} \\
% \mathtt{3x^2 \in L \subset Q} 
% \end{align*}


% potencial \verb$\psi$ $\psi$

% sigma \verb$\sigma$ $\sigma$

\section{Introducción}
En este trabajo se caracterizará estadísticamente un tren de disparos (potencial de acción) para una red neuronal del tipo Integración y Disparo con Perdida de memoria (IDP). Una red neuronal de este tipo está constituida por neuronas individuales e independientes interconectadas entre si, entrada-salida, con la capacidad de generar potenciales de acción (PA) en respuesta a los estímulos proveniente del resto de las neuronas de la red. El potencial de acción es el estado de activación de una neurona. Una neurona se modela matemáticamente en tiempo discreto como un sumador de los estados de la entrada, donde un valor fijo denominado umbral determina cuándo una neurona desarrolla un PA. Las neuronas están interconectadas entre sí mediante pesos sinápticos que regulan cuánto afecta el PA de una neurona sobre el estado de otra neurona. Las interacciones entre las neuronas describen la dinámica de la red neuronal y de esta forma si las interacciones no presentan ruido, el estado futuro de la red podría ser perfectamente determinado en base a los estados anteriores. Biológicamente, las secuencias de potenciales de acción representa información almacenada en la red, una conexión estímulo-respuesta que en algún tiempo pasado fue aprendido o heredado. Sin embargo, los registros obtenidos bajo las mismas condiciones experimentales no describen secuencias exactas de disparos, demandando con ello la búsqueda de regularidades estadísticas para la caracterización de una red neuronal real. Un forma de acercarse al comportamiento biológico mediante el modelo de IDP es introduciendo ruido en el modelo. 

Uno de los principales obstáculos para caracterizar una red neuronal bajo una secuencia de estímulos es la necesidad de conocer el estado inicial. Considerando que una red neuronal no comienza en el momento del registro sino al comienzo del desarrollo de la vida, es estado inicial tendrá una distribución probabilidad desconocida.

En este trabajo se presentará una caracterización completa del tren de disparos utilizando un modelo de IDP con ruido gaussiano en tiempo discreto y un estímulo externo constante. El trabajo se organiza de la siguiente forma: Probabilidad de que la neurona no dispare a tiempo $t$ dado el pasado, Probabilidad de transición no Markoviana, Existencia de una medida única e invariante, Aproximación Markoviana con memoria a $R$ estados y aplicaciones en neurociencia. 

\section{Definiciones y resultados preliminares}

\subsection{Modelo}

Sea una red definida por:

\begin{itemize}
    \item $N$ dimensión de la red neuronal (cantidad de neuronas).
    \item W matriz de pesos sinápticos. $W_{ij}$ intensidad de asociación entre las neurona $i$ con la $j$. Los pesos sinápticos son acotados $W_{ij} \in [W_{min},W_{max}]$
    \item $V=(V_i)_{i=1}^N$ vector de potencial de membrana, donde cada $V_i$ representa el potencial de membrana de la neurona $i$ con $i={1,...,N}$.
    \item $\theta$ umbral de disparo, invariante respecto del tiempo y único para todas las neuronas, donde $\theta \in \reals$ y $\theta >0$.
    \item $Z(x)$ una función indicadora que indica si la neurona disparó.
\end{itemize}

Definimos $Z(V_i) \in \reals$ como:

\begin{equation*}
    Z(V_i) :=  \left\{ \begin{array}{ll}
            1   & \text{si el potencial de la neurona } i \text{ supera el umbral}, \ V_i > \theta \text{ , } i={1,..,N} \\ 
            0  & \text{otros casos}
            \end{array}\right.
    \label{eqn:indicadoraV}
\end{equation*}
es decir, $Z(V_i)=\indi{V_i>\theta}$. Definimos $Z(V)= (Z(V_i))^N_{i=1}$ como el vector de estados de la red.

\subsubsection{Dinámica de la red}

Trabajaremos con tiempo $t$ discreto, y en esta primera parte consideraremos la que la red comienza en un estado inicial de potenciales de acción $V(s)$ donde $s \in \enteros$ y acotado. 
Sea $\gamma \in [0,1]$ la tasa de olvido de la red (\emph{leak rate}), los potenciales de membrana se comportarán de acuerdo a la siguiente dinámica, donde los parámetros $W$, $\gamma$, $N$ y $\theta$ son constantes para todo tiempo discreto:

\begin{equation}
    V(t+1) = F(V(t)) + \sigma_B B(t)
    \label{eqn:dinamica1}
\end{equation}
donde $F(V(t)) = \left(F_i(V(t))\right)_i$ y
\begin{equation}
    F_i(V(t)) = \gamma\ V_i(t)\ (1-Z(V_j(t))) + \sum_{j=1}^N W_{ij} Z(V_i(t)) +I_i 
    \label{eqn:dinamica2}
\end{equation}

El término $I_i$ representa el campo externo o contribuciones externas al sistema. $I_i \in \reals$ y en este trabajo consideraremos que $I_i$ son acotados y constantes para todos tiempo discreto.

\subsubsection{Ruido}

El ruido está representado a través de $B(t)=(B_i(t))^N_{i=1}$ en el segundo término de la ecuación \eqref{eqn:dinamica1}. Cada $B_i(t)\sim\mathcal{N}(0,1)$ representa ruido aditivo gaussiano y se consideran independientes las neuronas. El factor $\sigma_B$ es un factor de ajuste para la amplitud del ruido.

Definimos:
\begin{equation}
    \Pi(x) := \frac{1}{\sqrt{2\pi}} \int_x^{+\infty} e^{-\frac{u^2}{2}} du
\end{equation}
como la función de distribución acumulada a izquierda de $B_i(t)$ que, por su definición, representa  $\Pi(x)=\prob{B_i(t)>x}$.

Introducimos la normalización de una variable aleatoria, que será de utilidad mas adelante:
\begin{equation}
    \prob{X>x_0} = \prob{\frac{X-\mu}{\sigma}>\frac{x_o-\mu}{\sigma}} = \Pi\left(\frac{x_o-\mu}{\sigma}\right)
    \label{eqn:normgauss}
\end{equation}

\subsection{Definiciones}

\subsubsection{Secuencia de disparo}

Sea $\mathcal{M}=\reals^N$ un espacio de fase para nuestro sistema. Dos números $s,t \in \enteros$, $s<t$ tal que $V_s^t:=~\!\!(V(s),...,V(t))$ es una trayectoria de potenciales de acción de la red entre los tiempos $s$ y $t$. Cada $V_i(t)$ se asocia a un $\omega_i(t) = Z(V_i(t))$. Definimos como patrón de disparo de la red neuronal a tiempo $t$, al vector $\omega(t)=(\omega_i(t))^N_{i=1}$, que indica qué neurona disparo a tiempo $t$, $\omega_i(t)=1$.

Definimos un bloque de disparo como la secuencias de patrones de disparos entre los instantes $t$ y $s$, $\omega_s^t=[\omega(s),...,\omega(t)]$. Además la concatenación entre dos bloques de disparos se define como $\omega_s^{t_1}\omega_{t_1}^t=\omega_s^t$ con $s<t_1<t \in \enteros$.

\subsubsection{Configuraciones (\emph{Raster Plot})}

Definimos $\mathcal{A}$ (equiv. $\Omega_o$) el alfabeto, el set de los posibles patrones de disparos. $\mathcal{A}=\{0,1\}^N$ con $|\mathcal{A}|=2^N$. El espacio de configuraciones, $\mathcal{A}^{\enteros}$ tendrá como elementos los $\omega=\{\omega(t)\}^\infty_{t=-\infty}$ con $t \in \mathbb{Z}$. 

Sea $\sigma([\omega_s^t])$ una $\sigma$-álgebra constituida por un set de cilindros $[\omega_s^t]=\{ \omega'(n) \in \mathcal{A}^{\enteros} \text{ , } \omega'(n)=\omega(n) \text{ , } n=\{s,...,t\} \}$.   


\subsubsection{Distancia}
Definimos la función distancia en $\mathcal{A}^{\enteros}$ como:

\begin{equation}
    d_{\theta}(\omega,\omega') := \left\{ \begin{array}{ll}
        1 & \text{si } \omega \text{ y } \omega' \text{ difieren por primera vez en el instante } n\\
        0  & \text{si } \omega=\omega'
    \end{array}\right.
    \label{eqn:distancia}
\end{equation}

\subsubsection{Tiempo del último disparo}
Dada la secuencia $\omega_s^t$ con $s \text{ y } t \in \enteros / s<t$ y cada $i=1,...,N$, se define el tiempo de último disparo a la función:
\begin{equation}
    \tau_i(\omega_s^t) := \left\{ \begin{array}{ll}
        s   & \text{si  } \omega_i(k)=0, k=\{s,t\}. \\ 
        \text{max}\{s \leq k \leq t, \omega_i(k)=1\} & \text{si  } \exists k \in \{s,...,t\} \text{ tal que }\omega_i(k)=1,
    \end{array}\right.
    \label{eqn:ultimoDisparo}
\end{equation}

Si bien llamamos ``tiempo del último disparo'' a $\tau_i$, podría ser el caso en que $\tau_i(\omega_s^t)=s$, debido a que NO hubo disparos, y es una situación indistinguible de que haya habido disparo en tiempo $s$.
En este caso decimos que
\begin{equation*}
    \nexists  k \in [s+1,t] / \omega_i(k)=1 \left\{ \begin{array}{ll}
        \text{si hay disparo en tiempo }s   & \tau_i(\omega_s^t)=s \text{ y }  \omega_i(s) = 1 \\ 
        \text{si no hay disparo en tiempo }s   & \tau_i(\omega_s^t)=s \text{ y }  \omega_i(s) = 0
    \end{array}\right.
\end{equation*}

\subsection{Distribución de probabilidad del potencial de membrana}

Sea $P = \mathcal{N}(0,1)^{N \times \enteros}$ la distribución de probabilidad conjunta de la trayectoria del ruido para las $N$ neuronas para todo instantes $t\in \enteros$. En $P$ el potencial de membrana es un proceso estocástico y su evolución está dada por la ecuación \eqref{eqn:dinamica1}.

\subsubsection{Distribución de probabilidad condicional de $V(t+1)$}

Dado el par $(s,t)$, los cilindros con base $[\omega_s^t]=\mathcal{C}(\omega_s^t)$ forman una base de la $\sigma$-álgebra en $\mathcal{A}^{\enteros}$.
Primero analizaremos la distribución de probabilidad de $V(t+1)$ condicionada a $\omega_s^t=Z(V_s^t)$ y una condición inicial $V(s)$ que consideraremos acotada.

\paragraph{Proposición 1.} Para cada par (s,t), condicionado a la secuencia de disparos  $Z(V_s^t)=\omega_s^t$, es decir conocemos qué neuronas dispararon y cuándo lo hicieron, en el intervalo $[s,t]$ y, dado $V(s)$ el potencial en el instante inicial, tenemos que el potencial de membrana en el instante $t+1$ es:
\begin{equation}
    V(t+1) = \left\{ \begin{array}{ll}
        \gamma^{t+1-s} V(s) + C_i(\omega_s^t) + \sigma_B \  \xi_i(\omega_s^t)   & \text{si la neurona } i \text{ no disparó entre } s \text{ y } t\\ 
        C_i(\omega_s^t) + \sigma_B \ \xi_i(\omega_s^t)  & \text{en todo otro caso}
    \end{array}\right.
    \label{eqn:potencialDef}
\end{equation}
donde
\begin{equation}
     C_i(\omega_s^t) = \sum_{j=1}^N W_{ij} X_{ij}(\omega_s^t) + I_i \frac{1-\gamma^{t+1-\tau_i(\omega_s^t) }}{1-\gamma}
\end{equation}
\begin{equation}
    X_{ij}(\omega_s^t) = \sum_{l=\tau_i(\omega_s^t)}^t \gamma^{t-l} \omega_j(l)
    \label{eqn:Xij1}
\end{equation}
\begin{equation}
    \xi_i(\omega_s^t) = \sum_{l=\tau_i(\omega_s^t)}^t \gamma^{t-l} B_i(l)
\end{equation}

Esta proposición indica que la neurona pierde la memoria cuando dispara, o sea, la neurona vuelve a una estado basal de potencial de membrana y sólo los estados activos del resto de las neuronas de la red contribuyen al incremento de su potencial. 

\begin{proof}[\bf{Demostración de la Proposición 1}]
Primero analizaremos el caso simple en el cual no hay disparos de ninguna neurona entre los tiempos $s$ y $t$, es decir, $\tau_i(\omega_s^t)=s$ y $\omega_i(k)=0, \forall k \in [s,t]$. En esta condición el término $(1-\omega_i(k))$ es siempre $1$. Ademas no consideramos el factor estocástico $B_i$. Partimos de las ecuaciones \eqref{eqn:dinamica1} y \eqref{eqn:dinamica2} para determinar el potencial de membrana de la neurona $i$
\begin{align*}
    V_i(s+1) &= \gamma V_i(s)(1-\omega_i(s)) +I_i = \gamma V_i(s) +I_i\\
    V_i(s+2) &= \gamma V_i(s+1)(1-\omega_i(s+1)) +I_i = \gamma^2 V_i(s) + (\gamma + 1) I_i \\
    V_i(s+3) &= \ldots = \gamma^3 V_i(s) + (\gamma^2+\gamma + 1) I_i
\end{align*}

Continuando con la sucesión, si evaluamos en el instante $s + (t+1-s)$ tenemos que:
\begin{equation*}
    V_i(s+(t+1-s)) = V_i(t+1) = \gamma^{t+1-s} V_i(s)+(\gamma^{t+s} +...+\gamma+1)
\end{equation*}
\begin{equation}
    V_i(t+1) =\gamma^{t+1-s}V_i(s)+\frac{1-\gamma^{t+1-s}}{1-\gamma} I_i
    \label{eqn:potencial1}
\end{equation}

Si hubo uno o más disparos de la neurona $i$ entre los tiempos $s$ y $t$, siendo el último en el instante $k$ tenemos que: $\tau_i(\omega_s^t)=k$, $\omega_i(k)=1$ y $\omega_i(k+1)=0,...,\omega_i(t)=0$, donde tendremos que $(1-\omega_i(k))=0$. Entonces, más allá de lo que suceda entre los instantes $s$ y $k$, podemos calcular el potencial en $k+1$, a partir de las ecuaciones \eqref{eqn:dinamica1} y \eqref{eqn:dinamica2}, como:
\begin{equation*}
    V_i(k+1) &= \gamma V_i(k)(1-\omega_i(k)) + I_i = I_1 \\
\end{equation*}
donde hemos aplicado que $1-\omega_i(k)=0$. A partir de este último disparo, sabiendo que ya no hay un nuevo disparo entre $k+1$ y $t$, podemos calcular el potencial en los siguientes instantes como: 
\begin{align*}
    V_i(k+2) &= \gamma V_i(k+1)(1-\omega_i(k+1)) +I_i = (\gamma+1) I_i \\
    V_i(k+3) &= \gamma V_i(k+2)(1-\omega_i(k+2)) +I_i = (\gamma) (\gamma+1) I_i + I_i\\
             &= (\gamma^2+\gamma + 1) I_i 
\end{align*}

Continuando con la sucesión llegamos a:
\begin{equation}
    V_i(t+1) = \frac{1-\gamma^{t+1-\tau_i(\omega_s^t)}}{1-\gamma} I_i
    \label{eqn:potencial2}
\end{equation}
que resulta la misma es la misma expresión que \eqref{eqn:potencial1} sin el término de la condición inicial y reemplazando $s$ por $\tau_i(\omega_s^t)$, es decir el tiempo del último disparo. Una vez que se produce el disparo de la neurona en algún instante entre $t$ y $s$, el instante inicial no tiene incidencia en el estado futuro de la neurona. Esto implica que luego de un disparo, la neurona ``pierde la memoria''.

Ahora analizamos el caso en que la neurona $i$ nunca disparó (es decir $\omega(k)=0 \ \forall \ k$, pero contemplando la incidencia del resto de las neuronas del sistema. Además tenemos en cuenta el factor estocástico $B_i$. 

Nuevamente, considerando el instante inicial $V_i(s)$ desarrollamos el potencial para los instantes posteriores a partir de las ecuaciones \eqref{eqn:dinamica1} y \eqref{eqn:dinamica2}:
\begin{align*}
    V_i(s+1) =\ & \gamma V_i(s)\ (1-\omega_i(s))+ \sum_{j=1}^N  W_{ij} \ \omega_j(s) + \sigma_B B_i(s) + I_i \\
    V_i(s+2) =\ & \gamma V_i(s+1)\ (1-\omega_i(s+1))+ \sum_{j=1}^N  W_{ij} \ \omega_j(s+1) +
               \sigma_B B_i(s+1) + I_i \\
             =\ & \gamma^2 V_i(s) + I_i(\gamma+1) + \sigma_B[\gamma B_i(s) + B_i(s+1)] + \left[\gamma \sum_{j=1}^N  W_{ij} \omega_j(s) + \sum_{j=1}^N  W_{ij} \ \omega_j(s+1) \right]\\
    V_i(s+3) =\ &\gamma^3 V_i(s) + I_i(\gamma^2+\gamma+1) + \sigma_B[\gamma^2 B_i(s) + \gamma B_i(s+1) + B_i(s+2)] + \\
    &\sum_{j=1}^N  W_{ij} \left[\gamma^2 \omega_j(s) + \gamma \omega_j(s+1) + \omega_j(s+2)\right]             
\end{align*}

Continuando con la sucesión, tenemos que:
\begin{align*}
    V_i(t+1) =\ & V_i(s+(t+1-s)) = \gamma^{t+1-s} V_i(s)+ \sigma_B [\gamma^{t-s} B_i(s) + \gamma^{t-s-1} B_i(s+1) + ...+\gamma B_i(t)] \\
                & +I_i(\gamma^{t-s}+ ... + \gamma +1) + \sum_{j=1}^N W_{ij} \left[\gamma^{t-s} \omega_j(s) + \gamma^{t-s-1} \omega_j(s+1) + ...+\gamma \omega_j(t-1) + \omega_j(t)\right]
\end{align*}
\begin{equation}
    V_i(t+1) =\ \gamma^{t+1-s}V_i(s)+ \sum_{j=1}^N W_{ij} \left(\sum_{l=s}^t \gamma^{t-l}\omega_j(l) \right) + \sigma_B \left(\sum_{l=s}^t \gamma^{t-l} B_i(l)\right) +\frac{1-\gamma^{t+1-s}}{1-\gamma} I_i
    \label{eqn:pruebaprop1a}
\end{equation}
En este caso, obtuvimos un resultado análogo a la ecuación \eqref{eqn:potencial1}, donde se agregaron los aportes de todas las neuronas de la red, más el aporte del ruido estocástico.

Finalmente, si hubo uno o más disparos entre los instantes $s$ y $t$, y en analogía con el desarrollo de \eqref{eqn:potencial2}, el término asociado a la condición inicial desaparece, y el aporte de las otras neuronas así como el ruido estocástico y el campo externo, solo influyen desde el instante del último disparo $\tau_i(\omega^t_s)$:
\begin{equation}
     V_i(t+1) = \sum_{j=1}^N W_{ij} \left(\sum_{l=\tau_i(\omega_s^t)}^t \gamma^{t-l}\omega_j(l) \right) + \sigma_B \left(\sum_{l=\tau_i(\omega_s^t)}^t \gamma^{t-l} B_i(l) \right) +\frac{1-\gamma^{t+1-\tau_i(\omega_s^t)}}{1-\gamma} I_i
     \label{eqn:potencial3}
\end{equation}
Con las ecuaciones \eqref{eqn:pruebaprop1a} y \eqref{eqn:potencial3}, la proposición 1 queda demostrada.
\end{proof}

\subsubsection{Esperanza Condicional de V(t+1)}

Dada una condición inicial y una secuencia $\omega_s^t$, el potencial de membrana \eqref{eqn:potencialDef} tiene una parte estocástica y otra determinística.
\begin{equation}
  V_i(t+1) = \underbrace{\gamma^{t+1-s} V_i(s) + C_i(\omega_s^t)}_{\text{determinístico}} + \underbrace{\sigma_B \  \xi_i(\omega_s^t)}_{\text{estocástico}}
  \label{eqn:potencialCompacto}
\end{equation}

Como los $B_i(t)$ son independiente con idéntica distribución (\emph{i.i.d.}), $\mathcal{N}(0,1)$, con $P$ la distribución de probabilidad conjunta, el término $ \xi_i(\omega_s^t)$ es una suma ponderada de variables aleatorias gaussianas, e independientes entre los distintos $i=1..N$.
\begin{equation}
\xi_i(\omega_s^t) = \sum_{l=\tau_i(\omega_s^t)}^t \gamma^{t-l} B_i(l)
\label{eqn:xi1}
\end{equation}

\paragraph{Proposición 2.}
Para cada par $(s,t)$, condicionado a $Z(V_s^t)=\omega_s^t$ la secuncia de disparos y $V(s)$ la condición inicial, con $P$ la distribución de probabilidad conjunta de $B_i(k)$, $k \in \mathbb{Z}$ y $V(t+1)$ con distribución gaussiana, la esperanza condicional del potencial de membrana y la matriz de covarianza son:
\begin{equation}
    \esperanza{V_i(t+1)|\omega_s^t, V(s)} = \left\{ \begin{array}{cl}
                        \gamma^{t+1-s} V_i(s) + C_i(\omega_s^t)   & \text{si la neurona $i$ no disparó entre $s$ y $t$.}\\ 
                         C_i(\omega_s^t)    & \text{en otro caso (sí hubo disparo)}
                    \end{array}\right.
    \label{eqn:espCond1}
\end{equation}
\begin{equation}
    \var{V_i(t+1)|\omega_s^t, V(s)} = \frac{1-\gamma^{2(t+1-\tau_i(\omega_s^t))}}{1-\gamma^2} := \sigma_i^2(\omega_s^t)
    \label{eqn:varianzapot}
\end{equation}
\begin{equation}
    \cov{V_i(t+1),V_j(t+1)|\omega_s^t, V(s)} = \sigma_i^2(\omega_s^t) \delta_{ij} = \left\{ \begin{array}{cl}
                        \sigma_i^2(\omega_s^t)   & \text{si } i=j\\ 
                         0    & \text{si } i\neq j
                    \end{array}\right.
\end{equation}

\begin{proof}[\bf{Demostración de la Proposición 2}]
Podemos calcular la esperanza y la varianza de $\xi_i(\omega_s^t)$ a partir de la ecuación \eqref{eqn:xi1}:
\begin{equation*}
    \esperanza{\xi_i(\omega_s^t)} = \sum_{l=\tau_i(\omega_s^t)}^t \gamma^{t-l} \  \underbrace{\esperanza{B_i(l)}}_{=0}=0
\end{equation*}
\begin{equation*}
    \var{\xi_i(\omega_s^t)} = \sum_{l=\tau_i(\omega_s^t)}^t \left(\gamma^{t-l}\right)^2 \underbrace{\var{B_i(l)}}_{=1} = \frac{1-\gamma^{2(t+1-\tau_i(\omega_s^t))}}{1-\gamma^2}
\end{equation*}

Entonces, a partir de la ecuación \eqref{eqn:potencialCompacto} y teniendo en cuenta estos resultados, se desprende la expresión de la esperanza condicional \eqref{eqn:espCond1} y la varianza del potencial de membrana \eqref{eqn:varianzapot}.

Finalmente, fijada la secuencia de disparos $\omega_s^t$ y la condición inicial $V(s)$, el potencial de membrana de cada neurona es determinístico salvo por el término $\sigma_B \  \xi_i(\omega_s^t)$ que como dijimos, es independiente de otras neuronas.
En consecuencia podemos afirmar que los potenciales de membrana $V_i(t+1)$ y $V_j(t+1)$ son condicionalmente independientes entre sí. Por lo tanto:
\begin{equation}
    \cov{V_i(t+1),V_j(t+1)|\omega_s^t, V(s)} = \sigma_i^2(\omega_s^t) \delta_{ij} = \left\{ \begin{array}{cl}
            \sigma_i^2(\omega_s^t)   & \text{si } i=j\\ 
             0    & \text{si } i\neq j
        \end{array}\right.
\end{equation}
donde $\delta_{ij}$ es la \emph{delta de Kronecker}. Esto completa la demostración de la proposición 2.
\end{proof}

\subsection{Probabilidad de disparo}

Primero vamos a analizar la probabilidad de que la neurona $i$ no dispare entre los instantes $s$ y $t$. Esta es la probabilidad de que el potencial $V_i(k) < \theta$ con $k \in [s,t]$ ($\theta$ es el umbral de disparo).

\begin{align*}
    \prob{ \bigcap_{n=s}^t \{ V_i(n) < \theta \} } & =
    \sum_{\omega_s^t \in \mathcal{A}^{t-s}}
       \prob{ \bigcap_{n=s}^t \{ V_i(n) < \theta\} | \omega_s^t } \prob{\omega_s^t}\\
     & =     \sum_{\omega_s^t \in \mathcal{A}^{t-s}}
       \prob{ \{ V_i(s) < \theta \cap  V_i(s+1) < \theta \cap... \cap V_i(t) < \theta \} | \omega_s^t } \prob{\omega_s^t}
\end{align*}
donde hemos aplicado la fórmula de probabilidad total.

Aplicando la propiedad $\prob{A \cap B | C} = \prob{A | B \cap C}\cdot\prob{B|C}$, $t-s-1$ veces llegamos a:
\begin{align*}
    \prob{ \bigcap_{n=s}^t \{ V_i(n) < \theta \} } & = 
    \sum_{\omega_s^t \in \mathcal{A}^{t-s}} \left[
       \prob{ V_i(t) < \theta \left| V_i(t-1) < \theta \cap... \cap V_i(s) < \theta \cap \omega_s^t \right.} \cdot \right.\\
       & \qquad \qquad \cdot \prob{ V_i(t-1) < \theta \left| V_i(t-2) < \theta \cap... \cap V_i(s) < \theta \cap \omega_s^t  \right.} \cdot ... \cdot \\
       & \left. \qquad \qquad \cdot \prob{ V_i(s+1) < \theta \left| V_i(s) < \theta \cap \omega_s^t \right.} \cdot \prob{ V_i(s) < \theta \left| \omega_s^t \right.} \cdot \prob{\omega_s^t} \right] \\
       &= \sum_{\omega_s^t \in \mathcal{A}^{t-s}} \left\{ \left[ \prod_{n=s+1}^t \!\!
       \prob{ V_i(n) < \theta \left|  \bigcap_{l=s}^{n-1} \{ V_i(l) < \theta \} \cap \omega_s^t \right.} \right]\!\! \cdot \!\prob{ V_i(s) < \theta \left| \omega_s^t \right. } \cdot \prob{\omega_s^t} \right\}
\end{align*}

Dentro de la productoria, la secuencia $\omega_s^t$ solo es relevante hasta el instante $n-1$, mientras que el segundo factor, correspondiente a la probabilidad del potencial en el instante inicial, es independiente de la secuencia de disparos. Entonces:
\begin{equation}
    \prob{ \bigcap_{n=s}^t \{ V_i(n) < \theta \} } \! = \!\!\!
    \sum_{\omega_s^t \in \mathcal{A}^{t-s}} \! \left\{ \! \left[ \prod_{n=s+1}^t  \!\!
      \prob{ V_i(n) < \theta \left|  \bigcap_{l=s}^{n-1} \{ V_i(l) < \theta \} \cap \omega_s^{n-1} \! \! \right. } \! \right] \!\! \cdot \! \prob{ V_i(s) \! < \! \theta } \! \cdot\! \prob{\omega_s^t} \right\}
    \label{eqn:probdisparo}
\end{equation}
    
Analizamos el término dentro de la productoria
\begin{equation*}
     \prob{ V_i(n) < \theta \left|  \bigcap_{l=s}^{n-1} \{ V_i(l) < \theta \} \cap \omega_s^{n-1} \right. }
\end{equation*}

Este término representa la probabilidad de que el potencial en determinado instante sea menor al umbral, dado que en todos los otros instantes previos también fue menor al umbral y dada una secuencia de disparos determinada.
Esto se puede representar con la expresión del potencial de membrana para el caso donde la neurona no ha disparado (ecuación \eqref{eqn:potencialDef})
\begin{equation*}
     \prob{ V_i(n) < \theta \left|  \bigcap_{l=s}^{n-1} \{ V_i(l) < \theta \} \cap \omega_s^{n-1} \right. } =
    \prob{ \gamma^{n-s} V(s) + C_i(\omega_s^{n-1}) + \sigma_B \  \xi_i(\omega_s^{n-1}) < \theta }
\end{equation*}

Esta expresión puede calcularse como 1 menos la probabilidad de que el potencial sea mayor que el umbral, y luego aplicar la definición de la función $\Pi(x)$ (ecuación \eqref{eqn:normgauss})
\begin{align*}
     \prob{ V_i(n) < \theta \left|  \bigcap_{l=s}^{n-1} \{ V_i(l) < \theta \} \cap \omega_s^{n-1} \right.} & =  1-  \prob{ \gamma^{n-s} V(s) + C_i(\omega_s^{n-1}) + \sigma_B \  \xi_i(\omega_s^{n-1}) > \theta } \\
     & =  \Pi\left(\frac{\theta - \gamma^{n-s} V(s) - C_i(\omega_s^{n-1})}{\sigma_B \frac{\sqrt{1-\gamma^{2(n-s)}}}{1-\gamma^2}} \right)
\end{align*}

Podemos decir que, como $V_i(s)$ y $W_{ij}$ se asumen acotados, para cualquier par $(n,s)$ con $n>s$
\begin{equation*}
     0 < a <\prob{ V_i(n) < \theta \left|  \bigcap_{l=s}^{n-1} \{ V_i(l) < \theta \} \cap \omega_s^{n-1} \right. } <b<1
\end{equation*}

Por otro lado, 
\begin{equation*}
     0 < c <\prob{ V_i(s) < \theta \left|  \omega_s^s \right.} <d<1
\end{equation*}

Entonces definimos:
\begin{equation}
    \Pi_- = \min\{a,c\} \qquad \Pi_+ = \max\{b,d\}
\end{equation}
con
\begin{equation}
    \Pi_- > 0  \qquad \Pi_+ <1
\end{equation}


\paragraph{Proposición 3.}
Existe una probabilidad, distinta de cero, de que la neurona $i$ NO dispare, acotada por
\begin{equation}
    0 < \left( \Pi_- \right)^{t-s} < \prob{ \bigcap_{n=s}^t \{V_i(n) < \theta \} }  < \left( \Pi_+ \right)^{t-s} < 1
\end{equation}

\begin{proof}[\bf{Demostración de la Proposición 3.}]
Partimos de la ecuación \eqref{eqn:probdisparo}

\begin{equation*}
    \sum_{\omega_s^t \in \mathcal{A}^{t-s}} \left\{  \left[ \prod_{n=s+1}^t
       \underbrace{ \prob{ V_i(n) < \theta \left|  \bigcap_{l=s}^{n-1} \{ V_i(l) < \theta \} \cap \omega_s^{n-1} \right.} }_{\text{\normalsize \textcircled{a}}} \right] \cdot  \underbrace{\prob{ V_i(s) < \theta}}_{\textoCircle{b}} \cdot \prob{\omega_s^t} \right\} = \textoCircle{A}
\end{equation*}
Como los términos $\textoCircle{a}$ y $\textoCircle{b}$ tienen a $\Pi_-$ y $\Pi_+$ como cotas superior e inferior respectivamente, tenemos que:
\begin{equation*}
    \sum_{\omega_s^t \in \mathcal{A}^{t-s}} \left[ \left( \prod_{n=s+1}^t
       \Pi_- \right) \cdot  \Pi_- \cdot \prob{\omega_s^t} \right]
       < \textoCircle{A} <
    \sum_{\omega_s^t \in \mathcal{A}^{t-s}} \left[ \left( \prod_{n=s+1}^t 
       \Pi_+ \right) \cdot  \Pi_+ \cdot \prob{\omega_s^t} \right]
\end{equation*}
\begin{equation*}
    \left( \prod_{n=s+1}^t \Pi_- \right) \cdot  \Pi_- \cdot \underbrace{\sum_{\omega_s^t \in \mathcal{A}^{t-s}} \left[ \prob{\omega_s^t} \right]}_{1}
       < \textoCircle{A} <
    \left( \prod_{n=s+1}^t \Pi_+ \right) \cdot  \Pi_+ \cdot \underbrace{\sum_{\omega_s^t \in \mathcal{A}^{t-s}} \left[ \prob{\omega_s^t} \right]}_{1}
\end{equation*}

\begin{equation}
    \left(\Pi_-\right)^{t-s} < \textoCircle{A} <
    \left(\Pi_+ \right)^{t-s} 
\end{equation}
Como $\Pi_-$ y $\Pi_-$ tienen sus cotas, resulta
\begin{equation}
    0<\left(\Pi_-\right)^{t-s} < \textoCircle{A} <
    \left(\Pi_+ \right)^{t-s} <1
\end{equation}
donde queda demostrada la proposición 3.
\end{proof}

\subsection{Extensión $s \rightarrow -\infty$} %TODO Chequear subtitulo

El principal inconveniente es que la distribución de probabilidad de $V(s)$, depende del instante $s$ y es desconocida. 
Si bien se puede \emph{asumir} alguna distribución de probabilidad particular, no hay ningún elemento que nos permita justificarla. En cambio, buscaremos encontrar lo que se denomina \emph{régimen permanente}, tomando $s \rightarrow - \infty$.

Si bien no podemos asumir ninguna distribución de probabilidad para $V(s)$, podemos asumir que $V(s)$ será acotada (es decir, $\exists \, M / V_i(s) < M \ \forall \ i$) lo que es fisiológicamente plausible, ya que representa el potencial de acción de cada neurona en el instante $s$.

Redefinimos $\tau_i(\omega_{-\infty}^t)$ como:
\begin{equation}
    \tau_i\left(\omega_{-\infty}^t \right) := \left\{ 
      \begin{array}{cl}
        - \infty & \text{si } \omega_i(k)=0 \ \forall \ k \leq t \\
        \max\{ k \ / \ \omega_i(k)=1 \} & \text{todo otro caso}
      \end{array}
    \right.
\end{equation}

\paragraph{Proposición 4.}
Extendemos la proposición 2 cuando $s \rightarrow-\infty$. $V(t+1)$ condicionado a $\omega_{-\infty}^t$ es gaussiana con media
\begin{equation}
    \esperanza{V_i(t+1) | \omega_{-\infty}^t} = C_i(\omega_{-\infty}^t) =
    \sum_{j=1}^N W_{ij} X_{ij}(\omega_{-\infty}^t) + I_i \frac{1-\gamma^{t-\tau_i(\omega_{-\infty}^t)+1}}{1-\gamma}
\end{equation}
con
\begin{equation}
    X_{ij}(\omega_{-\infty}^t) = 
    \sum_{l=\tau_i(\omega_{-\infty}^t)}^t \gamma^{t-l} w_j(l)
\end{equation}
y covarianza
\begin{equation}
    \cov{V_i(t+1),V_j(t+1) | \omega_{-\infty}^t} = \sigma_i^2(\omega_{-\infty}^t) \delta_{ij} = \sigma_B^2 \frac{1-\gamma^{2(t-\tau_i(\omega_{-\infty}^t)+2)}}{1-\gamma^2} \delta_{ij}
\end{equation}
con $V_i(t+1)$ para $i=\{1,...,N\}$ condicionalmente independientes.

\begin{proof}[\bf{Demostración de la Proposición 4.}]
Partimos de la expresión \eqref{eqn:potencialCompacto} donde extendemos $s$ a $-\infty$. Como $V(s)$ es acotado el término asociado a la condición inicial tiende a cero, por lo cual nos queda:
\begin{equation}
  V_i(t+1) = C_i(\omega_{-\infty}^t) + \sigma_B \  \xi_i(\omega_{-\infty}^t)
  \label{eqn:potencial}
\end{equation}
con
\begin{equation}
     C_i(\omega_{-\infty}^t) = \sum_{j=1}^N W_{ij} X_{ij}(\omega_{-\infty}^t) + I_i \frac{1-\gamma^{t+1-\tau_i(\omega_{-\infty}^t) }}{1-\gamma}
     \label{eqn:Ci}
\end{equation}
\begin{equation}
    X_{ij}(\omega_{-\infty}^t) = \sum_{l=\tau_i(\omega_{-\infty}^t)}^t \gamma^{t-l} \omega_j(l)
    \label{eqn:Xij}
\end{equation}
\begin{equation}
    \xi_i(\omega_{-\infty}^t) = \sum_{l=\tau_i(\omega_{-\infty}^t)}^t \gamma^{t-l} B_i(l)
    \label{eqn:xi}
\end{equation}

Entonces, podemos calcular la esperanza condicional de $V_i(t+1)$ como:
\begin{equation*}
    \esperanza{V(t+1) | \omega_{-\infty}^t)} = \esperanza{C_i(\omega) + \xi(\omega) | \omega=\omega_{-\infty}^t }
\end{equation*}
donde $C_i(\omega_{-\infty}^t)$ es, dado $\omega_{-\infty}^t$, determinístico. Entonces:
\begin{equation}
    \esperanza{V(t+1) | \omega_{-\infty}^t)} = C_i(\omega_{-\infty}^t) + \sigma_B \esperanza{\xi(\omega_{-\infty}^t)}
    \label{eqn:esperanza1}
\end{equation}

Debemos analizar si esta esperanza condicional está bien definida. Al extender $s$ a $-\infty$ se pueden presentar dos casos. Si la neurona $i$ dispara en uno o más instantes siendo $n$ el último, tenemos que $\tau_i(\omega_{-\infty}^t) = n$. En este caso la sumatoria $X_{ij}(\omega_{-\infty}^t)$ (ecuación \eqref{eqn:Xij}) resulta ser una suma finita de términos acotados, por lo cual converge. Además, $\xi_i(\omega_{-\infty}^t)$ (ecuación \eqref{eqn:xi1}) es una suma de una cantidad finita de variables aleatorias gaussianas, por lo tanto es una nueva variable aleatoria gaussiana, con:
\begin{equation*}
    \esperanza{\xi_i(\omega_{-\infty}^t)} = \sum_{l=\tau_i(\omega_{-\infty}^t)}^t \gamma^{t-l} \  \underbrace{\esperanza{B_i(l)}}_{=0}=0
\end{equation*}
y la ecuación \eqref{eqn:esperanza1} queda bien definida. %TODO reescribir
Si por el contrario encontramos que $\tau_i(\omega_{-\infty}^t) = -\infty$, tenemos que analizar la convergencia de $X_{ij}(\omega_{-\infty}^t)$, para esto buscamos acotar la sumatoria
%\begin{equation*}
%     X_{ij}(\omega_s^t) = \sum_{l=\tau_i(\omega_s^t)}^t \gamma^{t-l} \omega_j(l) \rightarrow \left\{
%     \begin{array}{lcl}
%          \tau_i(\omega_{-\infty}^t) = n & \Rightarrow & X_{ij}(\omega_{-\infty}^t) =  \sum_{l=n}^t \gamma^{t-l} w_j(l) \\
%          \tau_i(\omega_{-\infty}^t) = -\infty& \Rightarrow & X_{ij}(\omega_{-\infty}^t) = \sum_{l=-\infty}^t \gamma^{t-l} w_j(l)
%     \end{array}
%\right.
%\end{equation*}
%En el segundo caso es necesario analizar la convergencia de la sumatoria.
\begin{equation*}
X_{ij}(\omega_{-\infty}^t) = \sum_{l=-\infty}^t \gamma^{t-l} w_j(l) \leq \sum_{l=-\infty}^t \gamma^{t-l} = 
\sum_{k=0}^\infty \gamma^k = \frac{1}{1-\gamma}
\end{equation*}
donde hemos considerado para establecer la desigualdad que $\omega_j(l)=1 \ \forall \ l$. Utilizando el cambio de variable $k=t-l$ y recordando que $\gamma<1$ logramos reescribir la sumatoria como una serie geométrica.
 
Además, la esperanza de $\xi_i(\omega_{-\infty}^t)$ en este segundo caso resulta: 
%De la misma forma, podemos analizar la esperanza y la varianza de $\xi_i(\omega_{-\infty}^t)$.
\begin{equation*}
    \esperanza{\xi_i(\omega_{-\infty}^t)} = \sum_{l=-\infty}^t \gamma^{t-l} \  \underbrace{\esperanza{B_i(l)}}_{=0}=0
\end{equation*}

Entonces, para ambos casos la esperanza condicional de $V(_it+1)$ dado $\omega_{-\infty}^t$ converge y es
\begin{equation}
    \esperanza{V_i(t+1) | \omega_{-\infty}^t} = C_i(\omega_{-\infty}^t) =
    \sum_{j=1}^N W_{ij} X_{ij}(\omega_{-\infty}^t) + I_i \frac{1-\gamma^{t-\tau_i(\omega_{-\infty}^t)+1}}{1-\gamma}
\end{equation}

Por otro lado, la varianza de $V_t(t+1)$ depende de $\xi(\omega_{-\infty}^t)$. En el caso en que $\tau_i(\omega_{-\infty}^t)$ es finito, tenemos que
\begin{equation*}
    \var{\xi_i(\omega_{-\infty}^t)} = \sum_{l=\tau_i(\omega_{-\infty}^t)}^t \left(\gamma^{t-l}\right)^2 \underbrace{\var{B_i(l)}}_{=1} = \frac{1-\gamma^{2(t+1-\tau_i(\omega_{-\infty}^t))}}{1-\gamma^2}
\end{equation*}
y para el caso donde $\tau_i(\omega_{-\infty}^t) = -\infty$ 
\begin{equation*}
    \var{\xi_i(\omega_{-\infty}^t)} = \sum_{l=-\infty}^t \left(\gamma^{t-l}\right)^2 \underbrace{\var{B_i(l)}}_{=1} = \sum_{k=0}^\infty \left(\gamma^k\right)^2 = \sum_{k=0}^\infty \left(\gamma^2\right)^k =\frac{1}{1-\gamma^2}
\end{equation*}
donde hemos aplicado el cambio de variable $k=t-l$ y la expresión de la serie geométrica.

Fijado $\omega_{-\infty}^t$, $V_i(t+1)$ y $V_j(t+1)$ son condicionalmente independientes, por lo tanto podemos generalizar la varianza de $V_i(t+1)$ y la covarianza entre $V_i(t+1)$ y $V_j(t+1)$, condicionadas a $\omega_{-\infty}^t$ como
\begin{equation}
    \var{V_i(t+1) | \omega_{-\infty}^t} = \var{C_i(\omega_{-\infty}^t) + \sigma_B \  \xi_i(\omega_{-\infty}^t)} = \var{C_i(\omega_{-\infty}^t)} + \var{ \sigma_B \  \xi_i(\omega_{-\infty}^t)} 
\end{equation}
\begin{equation}
    \var{V_i(t+1) | \omega_{-\infty}^t} = \sigma_B^2 \frac{1-\gamma^{2(t-\tau_i(\omega_{-\infty}^t)+2)}}{1-\gamma^2}
\end{equation}

\begin{equation}
    \cov{V_i(t+1),V_j(t+1) | \omega_{-\infty}^t} = \sigma_i^2(\omega_{-\infty}^t) \delta_{ij} = \sigma_B^2 \frac{1-\gamma^{2(t-\tau_i(\omega_{-\infty}^t)+2)}}{1-\gamma^2} \delta_{ij}
\end{equation}

Esto completa la demostración de la proposición 4.

\end{proof}

\subsection{Cotas}
Será útil encontrar cotas para $X_{ij}$, $C_i$ y $\sigma_i$. Primero analizamos las cotas para $X_{ij}$

\begin{equation*}
    X_{ij}(\omega_{-\infty}^t) = \sum_{l=\tau_i(\omega_{-\infty}^t)}^t \gamma^{t-l} \omega_j(l)
    %\label{eqn:Xij}
\end{equation*}  
como $\gamma$ es positivo, el mínimo para $X_{ij}(\omega_{-\infty}^t)$ lo encontaremos cuando $\omega_j(l)=0 \ \forall \ l \in \enteros / l \leq t$. Por otro lado, la cota máxima la encontraremos cuando $\omega_j(l)=0 \ \forall \ l \in \enteros / l \leq t$.
\begin{equation*}
    X_{ij}(\omega_{-\infty}^t) \leq 
    \sum_{l=\tau_i(\omega_{-\infty}^t)}^t \gamma^{t-l} =
    \sum_{k=0}^{t-\tau_i(\omega_{-\infty}^t)} \gamma^{k} =
    \frac{1-\gamma^{t-\tau_i(\omega_{-\infty}^t)+1}}{1-\gamma} \leq
    \frac{1}{1-\gamma}
    %\label{eqn:Xij}
\end{equation*}  
donde para obtener la última desigualdad, consideramos el caso donde $\tau_i(\omega_{-\infty}^t) = -\infty$.
Entonces, $X_{ij}(\omega_{-\infty}^t)$ está acotado según
\begin{equation}
    0 \leq X_{ij}(\omega_{-\infty}^t) \leq 
    \frac{1}{1-\gamma}
    \label{eqn:cotaXij}
\end{equation}

Ahora analizamos las cotas para $C_i(\omega_{-\infty}^t)$. Según la ecuación \eqref{eqn:Ci}, 
\begin{equation}
     C_i(\omega_{-\infty}^t) = \sum_{j=1}^N W_{ij} X_{ij}(\omega_{-\infty}^t) + I_i \frac{1-\gamma^{t+1-\tau_i(\omega_{-\infty}^t) }}{1-\gamma}
\end{equation}
Para hallar cotas mínimas y máximas, analizaremos ambos términos de la expresión por separado. El mínimo en el primer término lo hallaremos en el caso en que $X_{ij}$ es máximo cuando $W_{ij}$ es negativo, y $X_{ij}$ es mínimo cuando $W_{ij}$ es positivo (o cero). En foma análoga podemos hallar el máximo para el primer término

\begin{equation}
     \underbrace{\frac{1}{1-\gamma}\sum_{j=1, W_{ij}<0}^N W_{ij}}_{X_{ij}(\omega_{-\infty}^t) = \left\{ \begin{array}{ccc}
         \frac{1}{1-\gamma} & \text{si} & W_{ij} < 0 \\
         0 & \text{si} & W_{ij} \geq 0
     \end{array} \right.} \leq \quad \sum_{j=1}^N W_{ij} X_{ij}(\omega_{-\infty}^t) \leq \quad \underbrace{\frac{1}{1-\gamma}\sum_{j=1, W_{ij}>0}^N W_{ij}}_{X_{ij}(\omega_{-\infty}^t) = \left\{ \begin{array}{ccc}
         \frac{1}{1-\gamma} & \text{si} & W_{ij} > 0 \\
         0 & \text{si} & W_{ij} \leq 0
     \end{array} \right.}
\end{equation}



\paragraph{Proposición 5.}

\begin{proof}[\bf{Demostración de la Proposición 5.}]
\end{proof}






\subsection{Estacionariedad}

\paragraph{Proposición 6} Fijamos una secuencia $a_{-\infty}^0$, $a(-n) \in \mathcal{A}, n\geq 0 \ \forall \ t$
\begin{equation}
    \prob{\omega(t)=a(0) | \omega_{-\infty}^{t-1}=a_{-\infty}^{-1})} = 
    \prob{(\omega(0)=a(0) | \omega_{-\infty}^{-1}=a_{-\infty}^{-1})} = 
    \prob{(\omega(1)=a(0) | \omega_{-\infty}^{0}=a_{-\infty}^{-1})}
\end{equation}

Para llegar a este resultado primero desarrollamos $X_{ij}$ (ver ecuación \eqref{eqn:Xij})
\begin{equation*}
    X_{ij}(\omega_{-\infty}^{t-1}) = \sum_{l=\tau_i(\omega_{-\infty}^{t-1})}^{t-1} \gamma^{t-1-l} \omega_j(l)
\end{equation*}
si $\omega_{-\infty}^{t-1} = a_{-\infty}^{-1}$ entonces $\tau_i(\omega_{-\infty}^{t-1}) = t + \tau_i(a_{-\infty}^{-1})$
\begin{equation*}
    X_{ij}(\omega_{-\infty}^{t-1}) = 
    \sum_{l=t + \tau_i(a_{-\infty}^{-1})}^{t-1} \gamma^{t-1-l} \omega_j(l) = 
    \sum_{l'= \tau_i(a_{-\infty}^{-1})}^{-1} \gamma^{-1-l'} \omega_j(l'+t) =
\end{equation*}
donde $l'=l-t$. Si consideramos que $\omega_{-\infty}^{t-1} = a_{-\infty}^{-1}$ entonces $\omega(t-n)=a(-n) \Rightarrow \omega(t+l') = a(l')$
\begin{equation}
    X_{ij}(\omega_{-\infty}^{t-1}) = 
    \sum_{l'= \tau_i(a_{-\infty}^{-1})}^{-1} \gamma^{-1-l'} a_j(l') = X_{ij}(a_{-\infty}^{-1})
\end{equation}

A continuación desarrollamos $C_i$ (ver ecuación \eqref{eqn:Ci})
\begin{align}
\nonumber    C_i(\omega_{-\infty}^{t-1}) &= \sum_{j=1}^N W_{ij} X_{ij}(\omega_{-\infty}^{t-1}) + I_i \frac{1-\gamma^{t-\tau_i(\omega_{-\infty}^{t-1})}}{1-\gamma} \\
    &= \sum_{j=1}^N W_{ij} X_{ij}(a_{-\infty}^{-1}) + I_i \frac{1-\gamma^{t-\tau_i(a_{-\infty}^{-1})}}{1-\gamma} = C_i(a_{-\infty}^{-1})
\end{align}

Finalmente desarrollamos $\sigma_i^2$ (ver ecuación \eqref{eqn:sigma})
\begin{equation}
  \sigma_i(\omega_{-\infty}^{t-1})= \sigma_B^2 \frac{1-\gamma^{2(t-\tau_i(\omega_{-\infty}^{t-1}))}}{1-\gamma^2} = \sigma_B^2 \frac{1-\gamma^{-2 \tau_i(a_{-\infty}^{-1})}}{1-\gamma^2} = \sigma_i^2(a_{-\infty}^{-1})
\end{equation}

Es importante notar que estas propiedades se cumplen sólo porque $W_{ij}$, $\gamma$ y $I_i$ son independientes del tiempo. Con estas propiedades desarrolladas, analizamos la dependencia con $t$ de probabilidad de obtener un patrón $\omega(t)$ dada una historia determinada.

\begin{align}
    \nonumber    \prob{\omega(t)=a(0) | \omega_{-\infty}^{t-1}=a_{-\infty}^{-1}} &= \prod_{i=1}^N \omega_i(t) \Pi\left( \frac{\theta - C_i(\omega_{-\infty}^{t-1})}{\sigma_i(\omega_{-\infty}^{t-1})}\right)+(1-\omega_i(t)) \left[ 1-\Pi\left(\frac{\theta - C_i(\omega_{-\infty}^{t-1})}{\sigma_i(\omega_{-\infty}^{t-1})}\right)\right] \\
    \nonumber    & = \prod_{i=1}^N a_i(0) \Pi\left( \frac{\theta - C_i(a_{-\infty}^{-1})}{\sigma_i(a{-\infty}^{-1})}\right)+(1-a_i(0)) \left[ 1-\Pi\left(\frac{\theta - C_i(a_{-\infty}^{-1})}{\sigma_i(a_{-\infty}^{-1})}\right)\right] \\
    &=\prob{\omega(0)=a(0) | \omega_{-\infty}^{-1}=a_{-\infty}^{-1}}
\end{align}

Por lo cual podemos decir que es un proceso estacionario.


\section{Medida de probabilidad}

Como hemos probado que el sistema es estacionario, nos concentramos en $P(\omega(0)|w_{-\infty}^{-1})$. Trabajamos entonces con secuencias $\mathcal{A}_{-\infty}^{0}$.

A continuación utilizaremos la siguiente notación: $\omega=\omega_{-\infty}^0$ ; $\uom=\omega_{-\infty}^{-1}$ ; $\Omega=\mathcal{A}_{-\infty}^0$ ; $\uOm=\mathcal{A}_{-\infty}^{-1}$.

La $\sigma$-álgebra de cilindros en $\Omega$, $\sigma(\Omega)=\mathcal{F}$ y, análogamente, $\sigma(\uOm)=\underline{\mathcal{F}}$.

Se define la función $T$ como el desplazamiento a derecha sobre $\Omega$.

\begin{equation}
    \left. \begin{array}{rcl}
         T: \Omega & \rightarrow & \Omega  \\
         \omega(t) & \rightarrow & \omega(t-1) \  t \leq 0
    \end{array} \right\} (T\omega)(t) = \omega(t-1) \text{ con }  t\leq 0
\end{equation}
con $a \in \mathcal{A} \Longrightarrow \omega'=\omega a \ / \ \omega'(t-1) = \omega(t) \  t \leq 0  \  \omega'(0) = a$


\subsection{$g$-funciones} 

Las $g$-funciones son funciones medibles tales que:
\begin{equation}
    g: \Omega \rightarrow [0,1] \text{ tal que } \forall \ \omega \in \Omega :
    \sum_{\omega', T(\omega')=\omega} g(\omega') = 1 = \sum_{a\in\mathcal{A}} g(\omega a)
\end{equation}

Notemos que la función de probabilidad de transición es una $g$-función

\begin{equation}
    g_0(\omega) = \prob{\omega(0)|\uom} = 
    \prod_{i=1}^N \left[ \omega_i(0) \Pif - (1-\omega_i(0))\log\left(1-\Pif \right)\right]
    \label{eqn:gof}
\end{equation}

\begin{equation}
   \sum_{\omega', T(\omega')=\omega} g_0(\omega') = 
   \sum_{a \in \mathcal{A}} \prob{a|\omega} = 1
\end{equation}

Las $g$ funciones cumplen las siguientes propiedades:

a) Una $g$-función es no nula si para todo $\omega in \Omega$, $g(\omega)>0$.

b) La "variación" de una $g$-función se define como:
\begin{equation}
    var_k(g):= \sup_{\omega,\omega' \in \Omega} \left\{\abs{g(\omega)-g(\omega')}: \omega(t)=\omega'(t) \ \forall t \in \{-k, ... ,0\}\right\}
\end{equation}

c) Una $g$-función es continua si
\begin{equation}
    var_k(g) \rightarrow 0 \text{ cuando } k \rightarrow +\infty
\end{equation}

\paragraph{Proposición 7} La función $g_0(\omega)$ es no nula.

Demostración: Suponemos que existe $\omega \in \Omega$ tal que $g_0(w)=\prob{\omega(0)|\uom}=0$, entonces existe una neurona (o un sitio) para la cual $\Pif = 0$ ó $1$ (ver ecuación \eqref{eqn:gof}).
Para que esto se cumpla tendría que suceder que $C_i(\uom) = \pm \infty$ ó que $\sigma_i(\uom) = 0$. Pero estas magnitudes están acotadas según \eqref{eqn:cotaCi} por lo tanto $g_0(\omega) > 0 \ \forall \ \omega$


\paragraph{Proposición 8} $g_0(\omega)$ es continua.

Demostración: Para realizar la demostración de la proposición 8 utilizaremos las siguientes propiedades (demostradas en el anexo).

a) Dada una colección $a_i,b_i$ tal que $0\leq a_i\leq 1$ y $0 \leq b_i \leq 1$, $\forall \ i = \{1, ... \ , N\}$ 
\begin{equation}
    \abs{\prod_{i=1}^N a_i - \prod_{i=1}^N b_i} \leq \sum_{i=1}^N \abs{a_i - b_i}
\end{equation}

b) El desarrollo en serie de Taylor, de $\sqrt{1-x}$ cuando $0 \leq x \leq 1$ es:
\begin{equation}
    \sqrt{1-x}= 1-\sum_{n=1}^{+\infty} f_n \cdot x^n \quad \text{con} \quad f_n = \frac{(2n)!}{(n!)^2 (2n-1) 4^n} \geq 0
    \label{eqn:taylor}
\end{equation}

c) Si $u$ y $v$ son tales que $0 \leq u \leq 1 \ , \ 0 \leq v \leq 1$ y dados $A,B \in \reals$ 

\begin{equation}
    \abs{A \sqrt{a-u} - B \sqrt{1-v} } \leq \abs{A-B} + \sum_{n=1}^{\infty} f_n \left(\abs{A} u^n + \abs{B} v^n\right)
    \label{eqn:desigtriangconproductoria}
\end{equation}

Para simplificar la notación se denominará, con $i=\{1, ... \ , N\}$ y $\uom=\omega_{-\infty}^{-1}$ 

\begin{align*}
    y_i&=\frac{\theta-C_i(\uom)}{\sigma_i(\uom)} \quad , \quad C_i=C_i(\uom) \quad , \quad \sigma_i=\sigma_i(\uom)  \quad , \quad \tau_i=\tau_i(\uom) \\
    y_i'&=\frac{\theta-C_i(\uom')}{\sigma_i(\uom')} \quad , \quad C_i'=C_i(\uom') \quad , \quad \sigma_i'=\sigma_i(\uom')  \quad , \quad \tau_i'=\tau_i(\uom')
\end{align*}

\begin{align*}
a_i&=\omega_i(0)\Pi(y_i)+(1-\omega_i(0))(1-\Pi(y_i)) \\ 
b_i&=\omega_i'(0)\Pi(y_i')+(1-\omega_i'(0))(1-\Pi(y_i'))
\end{align*}

Para $k>0$, $\omega(0)=\omega'(0)$, entonces
\begin{equation}
    \abs{a_i-b_i}=\abs{\omega_i(0) (\Pi(y_i) - \Pi(y_i')) + (1-\omega_i(0)) (\Pi(y_i) - \Pi(y_i'))} = \abs{\Pi(y_i)-\Pi(y_i')}
    \label{eqn:ayb}
\end{equation}

Calcularemos la variación de la función $g_0$,

\begin{align*}
    var_k(g_0(w)) & = \sup_{\omega,\omega' \in \Omega} \left\{\abs{g_0(\omega)-g_0(\omega')}: \omega(t)=\omega'(t) \ \forall t \in \{-k, ... ,0\}\right\} \\
    & = \sup_{\omega,\omega' \in \Omega} \left\{\abs{\prod_{i=1}^N a_i - \prod_{i=1}^N b_i}: \omega(t)=\omega'(t) \ \forall t \in \{-k, ... ,0\}\right\} \\
    & \leq \sup_{\omega,\omega' \in \Omega} \left\{\sum_{i=1}^N \abs{a_i - b_i} : \omega(t)=\omega'(t) \ \forall t \in \{-k, ... ,0\}\right\} \\
    & \leq \sum_{i=1}^N \sup_{\omega,\omega' \in \Omega} \left\{\abs{a_i - b_i} : \omega(t)=\omega'(t) \ \forall t \in \{-k, ... ,0\}\right\} 
\end{align*}

Utilizando la ecuación \eqref{eqn:ayb} resulta:
\begin{equation}
    var_k(g_0(w)) \leq \sum_{i=1}^N \sup_{\omega,\omega' \in \Omega} \left\{ \abs{\Pi(y_i)-\Pi(y_i')} : \omega(t)=\omega'(t) \ \forall t \in \{-k, ... ,0\}\right\}
    \label{eqn:varg0}
\end{equation}

Para un $k>0$ fijo, cuando $\omega$ es tal que el último disparo sucedió en algún instante entre $-k$ y $0$, es decir que $\tau(\uom) \in \{-k, ... \ , 0 \}$ tenemos que $\Pi(y_i) = \Pi(y_i')$. Entonces el supremo se obtiene para las configuraciones $\omega$ y $\omega'$ que tienen $\tau_i,\tau_i' < -k$

Como $\gamma < 1$

\begin{equation}
  \gamma^{-\tau}<\gamma^k \longrightarrow 1-\gamma^{-\tau}> 1-\gamma^k \longrightarrow \frac{1}{1-\gamma^{-\tau}} < \frac{1}{1-\gamma^k}
  \label{eqn:desigualdadgamma}
\end{equation}

Recordando la expresión de la función $\Pi(x)$

\begin{equation}
    \Pi(x)=\frac{1}{\sqrt{2\pi}} \int_x^{+\infty} e^{-\frac{u^2}{2}} du \Longrightarrow \Pi'(x)=\frac{1}{\sqrt{2\pi}} e^{-\frac{x^2}{2}}
\end{equation}

Entonces $\norm{\Pi'}_{\infty} = 1/\sqrt{2 \pi}$

Aplicando la propiedad \ref{prop:5} (ecuación \eqref{eqn:prop5})
\begin{equation}
    \abs{\Pi(y_i) - \Pi(y_i')} \leq \abs{y_i - y_i'} \norm{\Pi'}_{\infty} = \abs{y_i - y_i'} \frac{1}{\sqrt{2 \pi}}
    \label{eqn:derivada}
\end{equation}

Desarrollamos a continuación el término $\abs{y_i - y_i'}$

\begin{equation*}
    \abs{y_i - y_i'} = \abs{\frac{\theta - C_i(\uom)}{\sigma_i(\uom)} - 
    \frac{\theta - C_i(\uom')}{\sigma_i(\uom')}}
    = \abs{\frac{\theta - C_i}{\sigma_i} - \frac{\theta - C_i'}{\sigma_i'}}
    = \abs{\theta\left( \frac{1}{\sigma_i} - \frac{1}{\sigma_i'} \right)  - \left( \frac{C_i}{\sigma_i} - \frac{C_i}{\sigma_i'} \right) }
\end{equation*}
\begin{equation}
    \abs{y_i - y_i'} \leq \theta \abs{\frac{1}{\sigma_i} - \frac{1}{\sigma_i'}} + 
    \abs {\frac{C_i}{\sigma_i} - \frac{C_i'}{\sigma_i'}}
    \label{eqn:diffys}
\end{equation}

    %&= \abs{\theta\left( \frac{1}{\sigma_i} - \frac{1}{\sigma_i'} \right)  - \left( \frac{C_i %\sigma_i' - C_i'\sigma_i}{\sigma_i\sigma_i'}  \right) } \leq
    %\theta \abs{\frac{1}{\sigma_i} - \frac{1}{\sigma_i'}} + \abs {\frac{C_i \sigma_i' - C_i' %\sigma_i} {\sigma_i\sigma_i'}}


Recordando la expresión de $\sigma_i$ de la ecuación \eqref{eqn:sigmai}, el primer término de esta ecuación resulta:
\begin{align}
\nonumber   \abs{\frac{1}{\sigma_i} - \frac{1}{\sigma_i'}} &= \frac{\sqrt{1-\gamma^2}}{\sigma_B}
    \abs{ \frac{1}{\sqrt{1-\gamma^{2\tau_i}}} - \frac{1}{\sqrt{1-\gamma^{2\tau'_i}}} } = \\
     &=\frac{\sqrt{1-\gamma^2}}{\sigma_B \sqrt{1-\gamma^{2\tau_i}} \sqrt{1-\gamma^{2\tau'_i}} }
    \abs{ \sqrt{1-\gamma^{2\tau'_i}} - \sqrt{1-\gamma^{2\tau_i} }}
    \label{eqn:diffsigma}
\end{align}

Por un lado, recordando \eqref{eqn:desigualdadgamma} tanto $\sqrt{1-\gamma^{2\tau'_i}}$ y $\sqrt{1-\gamma^{2\tau_i} }$ son menores que $\sqrt{1-\gamma^{2k} }$. 
Por otro lado, aplicando la propiedad de la ecuación \eqref{eqn:taylor} tenemos que:

\begin{align}
    \abs{ \sqrt{1-\gamma^{2\tau'_i}} - \sqrt{1-\gamma^{2\tau_i} } } & \leq
    \sum_{n=1}^\infty f_n \left[ \left( \gamma^{-2\tau'_i}\right)^n + 
    \left( \gamma^{-2\tau_i}\right)^n  \right] \leq
    \sum_{n=1}^\infty f_n \left[ \left( \gamma^{-2\tau'_i}\right)^n + 
    \left( \gamma^{-2\tau_i}\right)^n  \right] \\
    & \leq 2 \sum_{n=1}^\infty f_n \gamma^{2kn} \leq 2 \gamma^{k} \sum_{n=1}^\infty f_n \gamma^{2k(n-1)} 
\end{align}

Por lo tanto:
\begin{equation}
  \abs{\frac{1}{\sigma_i} - \frac{1}{\sigma_i'}} \leq \frac{2 \sqrt{1-\gamma^{2}}}{\sigma_B (1-\gamma^2k)} \sum_{n=1}^\infty f_n \gamma^{2kn} = \frac{2 \gamma^{2k} \sqrt{1-\gamma^{2}}}{\sigma_B (1-\gamma^2k)} \sum_{n=1}^\infty f_n \gamma^{2k(n-1)} =
  \frac{2 \gamma^{2k} \sqrt{1-\gamma^{2}}}{\sigma_B (1-\gamma^{2k})} S(\gamma)
  \label{eqn:diffinvsigma}
\end{equation}
donde definimos
\begin{equation}
    S(\gamma) = \sum_{n=1}^\infty f_n \gamma^{2k(n-1)} = \sum_{n=0}^\infty f_n \gamma^{2kn}
\end{equation}
y se cumple que 
\begin{equation}
    \lim_{k\rightarrow\infty} S(\gamma) = 0
    \label{eqn:limiteS}
\end{equation}

Para desarrollar el segundo término de la ecuación \eqref{eqn:diffys} utilizamos el mismo desarrollo de la ecuación \eqref{eqn:diffsigma}

\begin{align}
\nonumber  \abs {\frac{C_i}{\sigma_i} - \frac{C_i'}{\sigma_i'}} &=
    \frac{\sqrt{1-\gamma^2}}{\sigma_B}
    \abs{ \frac{C_i}{\sqrt{1-\gamma^{2\tau_i}}} - \frac{C_i'}{\sqrt{1-\gamma^{2\tau'_i}}} }=\\
\nonumber   &=\frac{\sqrt{1-\gamma^2}}{\sigma_B \sqrt{1-\gamma^{2\tau_i}}
    \sqrt{1-\gamma^{2\tau'_i}} }
    \abs{ C_i\sqrt{1-\gamma^{2\tau'_i}} - C_i'\sqrt{1-\gamma^{2\tau_i} }} \\
    & \leq  \frac{\sqrt{1-\gamma^2}}{\sigma_B \left(1-\gamma^{2k}\right)}
    \abs{ C_i\sqrt{1-\gamma^{2\tau'_i}} - C_i'\sqrt{1-\gamma^{2\tau_i} }}
\end{align}


Aplicando la propiedad \eqref{eqn:desigtriangconproductoria} 
\begin{align*}
    \abs {\frac{C_i}{\sigma_i} - \frac{C_i'}{\sigma_i'}}
    & \leq  \frac{\sqrt{1-\gamma^2}}{\sigma_B \left(1-\gamma^{2k}\right)} 
    \left[ \abs{ C_i - C_i' } + \sum_{n=1}^{\infty} f_n \left[ \left( \gamma^{-2\tau_i'}\right)^n\abs{C_i} + 
    \left( \gamma^{-2\tau_i}\right)^n\abs{C_i'} \right] \right] \\
    &\leq \frac{\sqrt{1-\gamma^2}}{\sigma_B \left(1-\gamma^{2k}\right)} 
    \left[ \abs{ C_i - C_i' } + \sum_{n=1}^{\infty} f_n \gamma^{2kn} \left( \abs{C_i} + \abs{C_i'} \right) \right]
    \end{align*}
donde consideramos que $\gamma^{-2\tau_i}$ y $\gamma^{-2\tau_i'}$ son ambos menores que $\gamma^{2k}$. También, si acotamos $\abs{C_i}$ y $\abs{C_i'}$ por medio de \eqref{eqn:cotaCi} obtenemos
\begin{equation}
    \abs {\frac{C_i}{\sigma_i} - \frac{C_i'}{\sigma_i'}}
     \leq \frac{\sqrt{1-\gamma^2}}{\sigma_B \left(1-\gamma^{2k}\right)} 
    \left[ \abs{ C_i - C_i' } + 2 \abs{C_i^+} \gamma^{2k} S(\gamma) \right] 
    \label{eqn:diffcispond}
\end{equation}

Utilizando la definición de $C_i$ y $C_i'$ \eqref{eqn:Cib} tenemos:
\begin{align}
\nonumber \abs{ C_i - C_i' } &= 
\abs{ \sum_{j=1}^N W_{ij} \left( \sum_{l=\tau_i}^{-1}\gamma^{-l-1} \omega_j(l) -\sum_{l=\tau_i'}^{-1} \gamma^{-l-1} \omega_j'(l) \right) + \frac{I_i}{1-\gamma} \left(\gamma^{-\tau_i} - \gamma^{-\tau_i'} \right) }\\
&\leq \sum_{j=1}^N \abs{W_{ij}} \abs{ \sum_{l=\tau_i}^{-1}\gamma^{-l-1} \omega_j(l) -\sum_{l=\tau_i'}^{-1} \gamma^{-l-1} \omega_j'(l) } + \frac{\abs{I_i}}{1-\gamma} \abs{\gamma^{-\tau_i} - \gamma^{-\tau_i'} }
\label{eqn:diffcis}
\end{align}

Por un lado
\begin{equation}
    \abs{\gamma^{-\tau_i} - \gamma^{-\tau_i'} } \leq \abs{\gamma^{-\tau_i}} + \abs{\gamma^{-\tau_i'} } \leq 2 \gamma^k
    \label{eqn:restasgammas}
\end{equation}

Por otro lado, como $\omega_j(l)=\omega_j'(l)$ para todo $l \in \{-k, ..., 0\}$, entonces
\begin{align*}
    \abs{ \sum_{l=\tau_i}^{-1}\gamma^{-l-1} \omega_j(l) -\sum_{l=\tau_i'}^{-1} \gamma^{-l-1} \omega_j'(l) } & = 
    \abs{ \sum_{l=\tau_i}^{-k-1}\gamma^{-l-1} \omega_j(l) -\sum_{l=\tau_i'}^{-k-1} \gamma^{-l-1} \omega_j'(l) }  \\
    & \leq\abs{ \sum_{l=\tau_i}^{-k-1}\gamma^{-l-1} \omega_j(l)} + \abs{\sum_{l=\tau_i'}^{-k-1} \gamma^{-l-1} \omega_j'(l) }
\end{align*}

Si suponemos que $\omega_j(l)=1$ para todo $l \leq -k-1$ 
\begin{equation}
    \abs{ \sum_{l=\tau_i}^{-1}\gamma^{-l-1} \omega_j(l) -\sum_{l=\tau_i'}^{-1} \gamma^{-l-1} \omega_j'(l) } \leq
    \abs{ \sum_{l=\tau_i}^{-k-1}\gamma^{-l-1}} + \abs{\sum_{l=\tau_i'}^{-k-1} \gamma^{-l-1} } 
    \leq  \frac{2\gamma^k}{1-\gamma}
    \label{eqn:sumasgammas}
\end{equation}
donde, considerando que $\tau_i \leq -k-1  \Rightarrow -\tau_i \geq k+1 \Rightarrow -\tau_i -k \geq +1$, hemos utilizado
\begin{equation}
    \sum_{l=\tau_i}^{-k-1}\gamma^{-l-1} = \sum_{m=k}^{-\tau_i-1}\gamma^{m} = \sum_{m=0}^{-\tau_i-1-k}\gamma^{m+k} =  \gamma^k \frac{1-\gamma^{-\tau_i-k}}{1-\gamma} \leq
    \gamma^k \frac{1}{1-\gamma}
\end{equation}


Entonces, usando las ecuaciones \eqref{eqn:restasgammas} y \eqref{eqn:sumasgammas}, en la ecuación \eqref{eqn:diffcis}, queda:

\begin{equation}
 \abs{ C_i - C_i' } \leq 
 \sum_{j=1}^N \abs{W_{ij}}  \frac{2\gamma^k}{1-\gamma} + \frac{\abs{I_i}}{1-\gamma} 2 \gamma^k =
 \frac{2 \gamma^k}{1-\gamma} \left(\abs{I_i} + \sum_{j=1}^N \abs{W_{ij}} \right)
\end{equation}

Reemplazamos esta ecuación en la ecuación \eqref{eqn:diffcispond}
\begin{align}
\nonumber \abs {\frac{C_i}{\sigma_i} - \frac{C_i'}{\sigma_i'}} &
    \leq \frac{\sqrt{1-\gamma^2}}{\sigma_B \left(1-\gamma^{2k}\right)} 
    \left[ \frac{2 \gamma^k }{1-\gamma} \left( \abs{I_i} + \sum_{j=1}^N \abs{W_{ij}} \right) + 2 \abs{C_i^+} \gamma^{2k} S(\gamma) \right] \\
    &\leq \frac{2 \gamma^k \sqrt{1-\gamma^2}}{\sigma_B \left(1-\gamma^{2k}\right)} 
    \left[ \frac{1 }{1-\gamma} \left( \abs{I_i} + \sum_{j=1}^N \abs{W_{ij}} \right) + \abs{C_i^+}  \gamma^k S(\gamma) \right]
    \label{eqn:diffcispond2}
\end{align}

Reemplazamos \eqref{eqn:diffinvsigma} y \eqref{eqn:diffcispond2} en \eqref{eqn:diffys}

\begin{align}
\nonumber  \abs{y_i - y_i'}  & \leq 
    \theta \frac{2 \gamma^{2k} \sqrt{1-\gamma^{2}}}{\sigma_B (1-\gamma^{2k})} S(\gamma) + 
    \frac{2 \gamma^k \sqrt{1-\gamma^2}}{\sigma_B \left(1-\gamma^{2k}\right)} 
    \left[ \frac{1 }{1-\gamma} \left( \abs{I_i} + \sum_{j=1}^N \abs{W_{ij}} \right) + \abs{C_i^+}  \gamma^k S(\gamma) \right] \\
    & \leq \frac{2 \gamma^k \sqrt{1-\gamma^2}}{\sigma_B \left(1-\gamma^{2k}\right)} 
    \left[ \frac{1 }{1-\gamma} \left( \abs{I_i} + \sum_{j=1}^N \abs{W_{ij}} \right) + \left(\abs{C_i^+} +\theta \right) \gamma^k S(\gamma) \right]    
    \label{eqn:diffys2}
\end{align}

 Utilizando las ecuaciones \eqref{eqn:diffys2} y \eqref{eqn:derivada} en la ecuación \eqref{eqn:varg0} llegamos a
  \begin{align}
     \nonumber var_k(g_0(w)) & \leq \sum_{i=1}^N \sup_{\omega,\omega' \in \Omega} \left\{ \abs{\Pi(y_i)-\Pi(y_i')} : \omega(t)=\omega'(t) \ \forall t \in \{-k, ... ,0\}\right\} \\
\nonumber & \leq \sum_{i=1}^N \left\{
    \sqrt{\frac{2}{\pi}}\frac{ \gamma^k \sqrt{1-\gamma^2}}{\sigma_B \left(1-\gamma^{2k}\right)} 
    \left[ \frac{1 }{1-\gamma} \left( \abs{I_i} + \sum_{j=1}^N \abs{W_{ij}} \right) + \left(\abs{C_i^+} +\theta \right) \gamma^k S(\gamma) \right] \right\} \\
\nonumber & \leq  
    \sqrt{\frac{2}{\pi}} \frac{  \sqrt{1-\gamma^2}}{\sigma_B \left(1-\gamma^{2k}\right)} 
    \left[ \frac{1 }{1-\gamma} \left( \sum_{i=1}^N \abs{I_i} + \sum_{i=1}^N \sum_{j=1}^N \abs{W_{ij}} \right) + \gamma^k \left( N \theta + \sum_{i=1}^N \abs{C_i^+}  \right)  S(\gamma) \right]    \gamma^k
    \label{eqn:variacion}
 \end{align}

Teniendo en cuenta la expresión \eqref{eqn:limiteS} se cumple que:

\begin{align*}
    \lim_{k\rightarrow\infty} {\sqrt{\frac{2}{\pi}} \frac{  \sqrt{1-\gamma^2}}{\sigma_B \left(1-\gamma^{2k}\right)} 
    \left[ \frac{1 }{1-\gamma} \left( \sum_{i=1}^N \abs{I_i} + \sum_{i=1}^N \sum_{j=1}^N \abs{W_{ij}} \right) + \gamma^k \left( N \theta + \sum_{i=1}^N \abs{C_i^+}  \right)  S(\gamma) \right]  } = \\
     = \sqrt{\frac{2}{\pi}} \frac{  \sqrt{1+\gamma}}{\sigma_B \sqrt{1-\gamma} }
    \left( \sum_{i=1}^N \abs{I_i} + \sum_{i=1}^N \sum_{j=1}^N \abs{W_{ij}} \right) = \kappa
\end{align*}

Entonces:
\begin{equation}
    var_k(g_0(w)) \xrightarrow[k\rightarrow \infty]{} 0
\end{equation}
y lo hace en forma similar a $\kappa \gamma^k$

Por lo tanto $g_0(\omega)$ es continua

\subsection{Medida de Gibbs}
Una medida de probabilidad $\mu$ en $M(\Omega, \mathcal{F})$ es una $g$-medida (medida de Gibbs) si \cite{keane_strongly_1972}

\begin{equation}
    \int_\Omega f(\omega) g(\omega a) \mu(d\omega) = \int_{\omega(0)=a} f(\omega) \mu(d\omega)
\end{equation}
$\forall \ a \in \mathcal{A}$, $g(\omega)$ una $g$-función, $\forall f(\omega)$ medible en $\underline{\mathcal{F}}$.

Como $g_0$ es continua, siempre existe una $g$-medida \cite{keane_strongly_1972}. Johansson y Oberg \cite{johansson_square_2003} demostraron que si $g_0$ es una $g$-función, continua y no nula, y cumple

\begin{equation}
    \sum_{k\geq 0} var^2_k[\log(g_o(\omega))] < \infty
    \label{eqn:jona}
\end{equation}
es decir, la suma converge, entonces existe una única meidda de Gibbs.

\paragraph{Teorema 1} El sistema tiene una única $g_0$-medida independientemente de los parámetros $W_{i,j} , I_i, \gamma, \theta$ con $i,j=1...N$

Demostración: se utiliza la ecuación \eqref{eqn:jona} para lo cual se analiza la variabilidad de $\log(g_0)$

\begin{align*}
    \log(g_0(\omega))  & = \log \left[ \prod_{i=1}^N \left[ \omega_i(0)\Pi(y_i) - (1-\omega_i(0))\log(1-\Pi(y_i))\right] \right] = \\
     &= \sum_{i=1}^N \log\left[ \omega_i(0) \Pi(y_i)+(1-\omega_i(0))(1-\Pi(y_i)) \right] = \\
     &= \sum_{i=1}^N \left[ \omega_i(0) \log\left[\Pi(y_i)\right]+(1-\omega_i(0))\log\left[1-\Pi(y_i) \right] \right] = 
\end{align*}

donde se tuvo en cuenta que $\omega_i$ es $0$ ó $1$.

\begin{align}
\nonumber    var_k[\log(g_o(\omega))] &= \sup \left\{  \abs{\log(g_0(\omega)) - \log(g_0(\omega'))}  \forall \omega,\omega' \in \Omega \ / \  \omega(t) = \omega'(t) \ \forall \ t \in \{-k,...,0\} \right\}\\
\nonumber    &=\sup_{\omega,\omega' \in \Omega} \left\{  \abs{\sum_{i=1}^N \left[ \omega_i(0) \log\left(\frac{\Pi(y_i)}{\Pi(y_i')}\right) - (1-\omega_i(0)) \log \left(\frac{1-\Pi(y_i)}{1-\Pi(y_i')} \right) \right]} \right\} \leq \\
    & \leq \sum_{i=1}^N \sup_{\omega,\omega' \in \Omega} \left\{  \abs{ \left[ \log\left(\frac{\Pi(y_i)}{\Pi(y_i')}\right) - \log\left(\frac{1-\Pi(y_i)}{1-\Pi(y_i')}\right) \right]} \right\}
    \label{eqn:varlog1}
\end{align}

Recordando que:
\begin{align*}
    y_i= \frac{\theta - C_i(\omega_{-\infty}^{-1})} {\sigma_i(\omega_{-\infty}^{-1})} = 
         \frac{\theta - C_i(\uom)} {\sigma_i(\uom)}
\end{align*}

Como $C_i(\omega_{-\infty}^t)$ y $\sigma_i^2(\omega_{-\infty}^t)$ están acotadas según \eqref{eqn:cotaCi} y \eqref{eqn:cotaSigma} respectivamente se obtiene

\begin{equation}
    \theta - C_i^+ \leq \theta - C_i(\uom) \leq \theta - C_i^- \Longrightarrow
    \frac{\theta - C_i^+}{\frac{\sigma_B}{\sqrt{1-\gamma^2}}} \leq \frac{\theta - C_i(\uom)}{\sigma_i(\omega)} \leq \frac{\theta - C_i^- \Longrightarrow }{\sigma_B}    
\end{equation}
\begin{equation}
    \sqrt{1-\gamma^2}~\frac{\theta - C_i^+}{\sigma_B} \leq y_i, y_i' \leq \frac{\theta - C_i^- }{\sigma_B}
\end{equation}

Tomamos 

\begin{equation*}
    a= \min_{i\in\{1,...,N\}} \left\{\sqrt{1-\gamma^2}~\frac{\theta - C_i^+}{\sigma_B} \right\}
\end{equation*}
\begin{equation*}
    b= \max_{i\in\{1,...,N\}} \left\{\frac{\theta - C_i^-}{\sigma_B} \right\}
\end{equation*}

Recordando la propiedad \eqref{eqn:derivadavelocidad2} tomamos:

\begin{equation}
    \abs{\log \frac{\Pi(y_i)}{\Pi(y_i')}} = \abs{\log \Pi(y_i)  -\log \Pi(y_i')} \leq \abs{y_i-y_i'} \norm{\left(\log \Pi(x)\right)'}_{[a,b]}
\end{equation}


La función $(\log \Pi(x))'$ es:
\begin{equation}
    (\log \Pi(x))' = \frac{\Pi'(x)}{\Pi(x)} = \frac{e^{-\frac{x^2}{2}}}{\displaystyle \int_x^{+\infty}e^{-\frac{u^2}{2}}du}
\end{equation}

Cómo esta función es monótona creciente, tenemos que:
\begin{equation}
    \norm{\left(\log \Pi(x)\right)'}_{[a,b]} = \frac{e^{-\frac{b^2}{2}}}{\displaystyle \int_b^{+\infty}e^{-\frac{u^2}{2}}du} = \Upsilon
\end{equation}
donde $b<\infty$

Por propiedad de la función $\Pi(x)$, tal que $1-\Pi(x)=\Pi(-x)$ tenemos también que:
\begin{equation}
    \abs{\log \frac{1-\Pi(y_i)}{1-\Pi(y_i')}} = 
    \abs{\log \frac{\Pi(-y_i)}{\Pi(-y_i')}}  \leq \abs{(-y_i)-(-y_i')} \norm{\left(\log \Pi(x)\right)'}_{[a,b]}
\end{equation}

Entonces, reemplazando en la ecuación \eqref{eqn:varlog1}
\begin{align*}
  var_k[\log(g_o(\omega))] & \leq \sum_{i=1}^N \sup_{\omega,\omega' \in \Omega} \left\{  \abs{ \left[ \log\left(\frac{\Pi(y_i)}{\Pi(y_i')}\right) - \log\left(\frac{1-\Pi(y_i)}{1-\Pi(y_i')}\right) \right]} \right\} \\
  & \leq \sum_{i=1}^N 2 \Upsilon \sup_{\omega,\omega' \in \Omega} \left\{  \abs{y_i-y_i'} \right\} = 2 \Upsilon  \sum_{i=1}^N \sup_{\omega,\omega' \in \Omega} \left\{  \abs{y_i-y_i'} \right\}
\end{align*}

Recordando la ecuación \eqref{eqn:diffys2} podemos decir que $var_k[\log(g_o(\omega))] \leq K' \gamma^k$ con $K'=2 \sqrt{2 \pi} \Upsilon$

Además,
\begin{equation*}
    \sum_{k\geq 0} var^2_k[\log(g_o(\omega))] \leq \sum_{k\geq 0} (K' \gamma^k)^2 =(K')^2 \sum_{k\geq 0} (\gamma^k)^2 = (K')^2 \frac{1}{1-\gamma} < \infty \  (CV)
\end{equation*}
porque $\abs{\gamma}<1$

Entonces por el teorema de Johansson y Oberg \cite{johansson_square_2003}, existe una única $g$-medida.


\section{Potencial Regular}


Una función $\psi$ es un potencial regular si es una función continua y cumple con \cite{keller_equilibrium_1998}

\begin{align*}
\left. \begin{array}{l}
     \psi : \Omega \to \reals  \\
     \psi \text{ es } C^0 \\ 
     \displaystyle \sum_{k\geq0} var_k(\psi) < \infty
\end{array} \right\}
\Longrightarrow \psi \text{ es un potencial complejo}
\end{align*}

\subsection{Entropía}

Dada una medida $\mu \in M_T(\Omega)$ donde $M_T(\Omega)$  es el set de medidas de probabilidad invariantes ante la transformación T, se define la entropía de la medida como:

\begin{equation}
    h(\mu) = \limsup{\frac{1}{n+1} \sum_{[\omega]_0^n} \mu([\omega]_0^n) \log(\mu([w]_0^n))}
    \label{eqn:entropia}
\end{equation}

donde $[w]_0^n$ son todos los cilindros de longitud $n+1$

\paragraph{Definición 5} Se denomina $\mu_\psi$, un estado de equilibro del potencial $\psi$, a la medida invariante respecto a $T$ sobre $\Omega$ tal que:

\begin{equation}
    P(\psi) = h(\mu_\psi) + \mu_\psi (\psi) = \sup_{}\mu \in M_T(\Omega) \left\{ h(\mu)+\mu(\psi)\right\}
    \label{eqn:presiontopo}
\end{equation}

La presión topológica es cero cuando el potencial $\psi$ está normalizado. Es el caso de un potencial $\psi$ como el que utilizaremos ya que será el logaritmo de una probabilidad condicional\cite{keller_equilibrium_1998}.

Si $\psi$ es un potencial regular, entonces los estados de equilibrios de $\psi$ son las $g$-medidas asociadas a una $g$-función continua. En nuestro caso, donde sabemos que la $g$-medida es única, la función $g_0$ se relaciona con el potencial según

\begin{equation}
    \psi(\omega) = \log[g_0(\omega)]
\end{equation}


\paragraph{Teorema 2} Para cualquier set de parámetros ($W_{i,j}$, $\gamma$, $\umbral$,
$\sigma_B$) el sistema tiene una $g$-medida única, y es un estado de equilibro
para el potencial $\psi=\log[g_0(\omega)]$



\begin{align*}
  \psi(\omega) & = \psi(\omega_{-\infty}^0)=\log[g_0(\omega)]= \\
  & = \log\left[\prod_{i=1}^N\left[\omega_i(0) \ \Pi\left(\frac{\theta-C_i(\uom)}{\sigma_i(\uom)}\right)+(1-\omega_i(0))\left(1-\Pi\left(\frac{\theta-C_i(\uom)}{\sigma_i(\uom)}\right)\right)\right]\right] = \\
  & = \sum_{i=1}^N \log\left[\omega_i(0) \ \Pi\left(\frac{\theta-C_i(\uom)}{\sigma_i(\uom)}\right)+(1-\omega_i(0))\left(1-\Pi\left(\frac{\theta-C_i(\uom)}{\sigma_i(\uom)}\right)\right)\right]
\end{align*}

como $\omega(0)$ es 0 o 1,

\begin{equation}
  \psi(\omega) = 
  \sum_{i=1}^N \left[\omega_i(0) \log\left[ \Pi\left(\frac{\theta-C_i(\uom)}{\sigma_i(\uom)}\right)\right]+(1-\omega_i(0))\log\left[1-\Pi\left(\frac{\theta-C_i(\uom)}{\sigma_i(\uom)}\right)\right]\right] =
  \label{potencailnu}
\end{equation}


\paragraph{Proposición 9} El potencial de membrana $V(t)$ es estacionario con densidad de probabilidad producto

\begin{equation*}
  \rho_V(v) = \prod_{i=1}^N \rho_{V_i}(v_i) 
\end{equation*}

donde

\begin{equation*}
  \rho_{V_i}(v_i) = \int_{\uOm} \frac{1}{\sqrt{2 \pi \sigma_i(\uom)}} e^{-\frac{1}{2}\left(\frac{v-C_i(\uom)}{\sigma_i(\uom)}\right)^2} \mu_\psi(d\uom)
\end{equation*}

donde su esperanza es $\mu_\psi[C_i(\uom)]$ y la varianza es $\mu_\psi[\sigma_i^2(\uom)]$


\subsection{Tasa de disparos}

\begin{equation*}
  r_i(\omega) = P(\omega_i(0)=1 | \uom) = \Pif
\end{equation*}


\begin{equation*}
  r_i = \mu_\psi(\omega_i(0)=1)
\end{equation*}

\begin{equation*}
  r_i = \mu_\psi (r_i(\omega))=\Pif
\end{equation*}



\subsection{Entropía}

Como el potencial $\psi$ está normalizado $P(\psi)=0=h(\mu_\psi)+\mu_\psi(\psi)$

\begin{align}
   h(\mu_\psi) & = - \sum_{i=1}^N \mu_\psi \left[ \omega_i(0) \log\left(\Pif\right) + (1-\omega_i(0)) \log\left( 1-\Pif\right) \right]  \nonumber \\
  & = - \sum_{i=1}^N \left[\mu_\psi(\omega_i(0)) \ \mu_\psi\left( \log r_i(\omega) \right)+ \mu_\psi(1-\omega_i(0)) \ \mu_\psi \left(\log\left( 1-r_i(\omega) \right) \right) \right] \nonumber\\
  & = - \sum_{i=1}^N \left[ r_i \ \mu_\psi\left( \log r_i(\omega) \right)+ (1-r_i) \ \mu_\psi \left(\log ( 1-r_i(\omega) ) \right) \right] \label{eqn:entro1}
\end{align}
  
\paragraph{Proposición 10} La entropía del estado de equilibrio es positiva para cualquier conjunto de parámetros del sistema.

Debido a que $\psi(\omega) < 0 \  \forall \  \omega \in \Omega$ entonces $\mu_\psi(\psi) < 0$. A partir de la ecuación \eqref{eqn:entro1} vemos que la entropía es siempre positiva.


\subsection{Divergencia de Kullback-Leibler}

Dadas dos medidas T-invariantes $(\mu,\nu)$ la diveregencia de Kullback-Leibler se define como

\begin{equation}
    d_{KL}(\mu,\nu) = \lim_{n\to\infty} \frac{1}{n+1} \sum_{[w]_0^n} \mu([w]_0^n) \log \left( \frac{\mu([w]_0^n)}{\nu([w]_0^n)} \right)
\end{equation}


Para una medida $\mu$ ergódica y $\mu_\psi$ un estado de gibbs del potencial $\psi$

\begin{equation}
    d_{KL}(\mu,\mu_\psi) = p(\psi)-\mu(\psi)-h(\psi) = h(\mu_\psi)+\mu_\psi(\psi)-(h(\mu)-\mu(\psi)) = (h(\mu_\psi) - h(\mu)) + (\mu_\psi(\psi) - \mu_(\psi))
\end{equation}

\subsection{Estados de Gibbs}

Los estados de equilibrio de un potencial regular son las medidas de Gibbs o estados de Gibbs \cite{keller_equilibrium_1998}. Además, existe una constante $C_\psi >0$ tal que

\begin{equation}
    0 \leq e^{-C_\psi} \leq \frac{\mu_\psi([\omega]^n_0)}{e^{-(n+1)p(\psi)} e^{\sum^n_{k=0}\psi(T^k\omega)}} \leq e^{C_\psi}
\end{equation}

Como el potencial $\psi$ está normalizado, $p(\psi)=0$.
Esto implica que la medida toma la forma 

\begin{equation}
    \mu_\psi([\omega]^n_0) ==\frac{e^{\sum^n_{k=0}\psi(T^k\omega)}}{Z_\psi^{n+1}(\omega)}
\end{equation}

donde $Z_\psi^{n+1}(\omega)$ es análoga a la función de partición aunque en este caso depende de $\omega$

\nocite{keane_strongly_1972,ledrappier_principe_1974,johansson_square_2003,cessac_discrete_2010}


\section{Aproximación de Rango Finito}

El problema de trabajar con las probabilidades de transición y su estado de equilibrio asociado es que depende de la historia hasta $\tau_i(\omega{-\infty}^{-1})$ cuando este no es acotado.

Queremos aproximar la probabilidad de transición $P(\omega(0)|\omega_{-\infty}^{-1})$ con $P(\omega(0)|\omega_{-R}^{-1})$, donde $\tau_i(\omega_{-\infty}^{-1})$ se reemplaza por $\tau_i(\omega_{-R}^{-1})$. 

Ahora, existen una cantidad finita de bloques de disparos o estados de configuraciones posibles, que podrán ser codificadas mediante un número entero:
\begin{equation}
    w=\sum_{i=1}^N \sum_{n=-R}^{-1} 2^{(i-1)+(n+R)N} \omega_i (n). \label{eqCode_w}
\end{equation}

De esta forma cada palabra constituye una estado de una cadena de Markov, donde existen $2^{NR}$ estados posibles:

\begin{equation}
    \Omega^R = \{ 0,....,2^{NR}-1\}
\end{equation}

\subsection{Matriz de transición de estados}

De la proposición 6, se sabe que las probabilidades de transición no dependen del tiempo. Es decir, la cadena es \textit{homogenea}. Entonces tenemos la matriz $\mathcal{L}^{(R)}$ de dimensión $2^{NR}$ x $2^{NR}$:

\begin{equation}
    \mathcal{L}^{(R)}_{w',w}  \doteq    \left\{ \begin{array}{ll}
                            P(\omega(0)|\omega^{-1}_{-R}) & \text{si } w' \backsim \omega^{-1}_{-R},~ w \backsim \omega^{0}_{-R+1}, \\ 
                            0,           & \text{otros casos}
                            \end{array}\right.
    \label{matriz_L}
\end{equation}

Podemos definir las transiciones 'legales' o permitidas mediante una matriz de incidencia:

\begin{equation}
    \mathcal{I}_{w',w}  = \left\{ \begin{array}{ll}
                            1, & \text{si } w' \rightarrow w \text{ si es una transición legal} \\ 
                            0,           & \text{otros casos}
                            \end{array}\right.
\end{equation}

$\mathcal{I}$ es primitiva, es decir $\exists m > 0$ tal que $\forall w',~w \in \Omega^R x \Omega^R$ donde $\mathcal{I}^m_{w',w}>0$.
En efecto, $\mathcal{I}^m_{\omega',\omega}>0$ significa que $\exists$ una configuración que contenga los bloques $w'$ y $w$, donde los primeros patrones de cada bloque estén separados por R transiciones. Lo cual confirma que en m pasos se puede pasar de $w'$ a $w$.

\subsection{Potencial de rango R+1}

Utilizando la misma representación que en \eqref{eqCode_w} para los bloques de tamaño R+1, cada bloque  $\omega^0_{-R}$ podemos asociarlo a una palabra W:

\begin{equation}
    W=\sum_{i=1}^N \sum_{n=-R}^{0} 2^{(i-1)+(n+R)N} \omega_i (n) \backsim \omega^0_{-R} .
    \label{configuraciones_W}
\end{equation}

y definimos,

\begin{equation*}
    \psi^{(R)}(W) = \sum^N_{i=1} [ \omega_i(0) log( \Pif ) + (1-\omega_i(0)) log(1-\Pif)]
\end{equation*}

$\psi^{(R)}(W)$ es el potencial de rango R+1, correspondiente a una aproximación del potencial \eqref{potencailnu} cuando la memoria tiene profundidad R. 
Entonces, la probabilidad de ransición entre una configuración $w'$ y otra $w$ será:

\begin{equation*}
    \mathcal{L}^{(R)}_{w',w}= e^{ \psi^{(R)}(w)} \mathcal{I}_{w',w}
\end{equation*}

Del álgebra de matrices, podemos decir que $\mathcal{L}^{(R)}$ también es primitiva y tiene un autovalor real "s", que es el de mayor módulo y por ser $\mathcal{L}^{(R)}$ una matriz de probabilidades, será igual a $s=1$.
Y como consecuencia de esto, la presión topológica será:

\begin{equation}
    P(\psi^{(R)})= log(s)=0
\end{equation}

Los autovalores a izquierda y derecha $l\mathcal{L}^{(R)}=sl$ y $\mathcal{L}^{(R)}r=sr$ con $r_i>0$, donde $i\in |2^{NR}|$ las enumeración de las posibles configuraciones, son la unica medida invariante de probabilidad que tiene la cadena de Markov, $\mu_\psi^R = lr)$.
De esta forma, se pude calcular la probabilidad de un bloque de disparo con una longitud arbitraria mediante:

\begin{equation}
    \mu_\psi^{(R)}([\omega]^{t+R}_s)=\mu_\psi^{(R)}(w(s)) \prod_{n=s}^{t-1} \mathcal{L}^{(R)}_{w(n+1)',w(n)}
\end{equation}

con $w(n) \backsim \omega^{n+R}_n $.

Se puede comprobar que $ mu_\psi^{(R)}$ es un estado de Gibbs \cite{keller_equilibrium_1998} y como la presión topológica se iguala a cero, la entropia será:

\begin{equation*}
    h(\mu_\psi^{(R)}) = -\mu_\psi^{(R)}(\psi^{(R)})
\end{equation*}

\subsection{Convergencia de la aproximación del potencial de rango finito}

Una forma de analizar cuan bien aproxima una medida de rango finito, primero comparamos los potenciales asociados:

\begin{equation*}
    || \psi - \psi^{(R)} ||_{\infty} \leq sup{ |\psi(\omega) - \psi(\omega')| : \omega,\omega' \in X, \omega(t)=\omega'(t), \forall t \in {-R,...,0}  } \triangleq var_R(\psi)
\end{equation*}

que desde el teorema 1, estará acotado por:

\begin{equation*}
    || \psi - \psi^{(R)} ||_{\infty} \leq K'\gamma^R
\end{equation*}

Como $\mu_\psi^{(R)}$ y $\mu_\psi$ son distribuciones de Gibbs, usamos la divergencia \textcolor{red}{KL} para estados de equilibrio se puede encontrar una cota para la distancia entre la medida en rango finito respecto a la de rango infinito:

\begin{equation*}
    d(\mu_{\psi^{(R)}}, \mu_\psi) = P(\psi) - \mu_{\psi^{(R)}}(\psi) - h(\mu_{\psi^{(R)}}) = \mu_{\psi^{(R)}}(\psi^{(R)}-\psi)
\end{equation*}

donde presión topológica es cero porque $\psi$ está normalizado y desde el teorema 1 encontramos que la distancia estará acotada con:
\begin{equation*}
     d(\mu_{\psi^{(R)}},  \mu_\psi) < K'\gamma^R
\end{equation*}

Lo que indica que la distancia decae exponencialmente rápido con una velocidad $gamma$.
Un consecuencia práctica de estos resultados es la posibilidad de aproximar $\psi$ a un potencial de rango $R$, suponiendo que la distancia entre las medidas es aproximadamente 1:

\begin{align*}
    1 & \approxeq K'\gamma^R \\
    \log(1) = 0 & \approxeq \log(K')+R\log(\gamma) \\
    R\log(\gamma) & \approxeq - \log(K') \\
    R & \approxeq - \frac{\log(K')}{\log(\gamma)}
\end{align*}
    
De esta forma se puede estimar el rango a elegir en función de los parámetros del sistema. 


\section{Estadística de una Configuración (Raster plot)}

El modelo estadístico introducido en el sección anterior, basado en una aproximación Markoviana, permite calcular explícitamente los indicadores clásicos utilizado en neurociencia.

Utilizando el vector $W$ presentado en ~\ref{configuraciones_W}, se puede definir el potencia $\psi^{R}$ en función de $W \sim \omega_{-R}^0$:

\begin{equation}
    \psi^{R}(W) = \sum^{L=2^{N(R+1)}}_{n=1} \alpha_n \chi_{n(W)} \equiv \psi_{\alpha}^{(R)}(W),
\end{equation}

Donde $\alpha_n = \psi^{(R)}(W_n)$ con $n=1...L$ y $L=2^{N(R+1)}$. Llamamos a $\chi_n(W)$ un función indicadora que será 1 si $W=W_n$ y 0 en otro caso. 

Luego, $e^{\alpha_n}$ es la probabilidad condicional $P(\omega(0)|\omega^{-1}_{-R})$. Si fijamos el pasado $\omega^{-1}_{-R})$, la suma de los $e^{\alpha_n}$ sobre todos los bloque $W_n$ que tienen pasado $\omega^{-1}_{-R})$ y $\omega(0)=1$ es la probabilidad de que la neurona i dispare dado el pasado. El producto de los elementos de la matriz $\mathcal{L}^{(R)}_{w',w}$, definida en \ref{matriz_L}, permite calcular la probabilidad de cierta secuencias de disparo dada una historia determinada.

\begin{itemize}
    \item R $=$ la respuesta de un subset de neuronas.
    \item S $=$ el estímulo provisto por el subset de neuronas correspondientes al pasado determinado.
\end{itemize}

Se puede calcular, a partir de la matriz $\mathcal{L}^{(R)}_{w',w}$ y dado que $P(S)$ es conocida (a partir de la medida invariante $\mu_{\psi^{(R)}}$, la probabilidad condicional $P(R|S)$. Lo que permite caracterizar el código neuronal de una red, donde los estímulos son trenes de disparos. 



\section{Anexo}

\paragraph{Propiedad 1}
 Dada una colección $a_i,b_i$ tal que $0\leq a_i\leq 1$ y $0 \leq b_i \leq 1$, $\forall \ i = \{1, ... \ , N\}$ 
\begin{equation}
    P(N) : \abs{\prod_{i=1}^N a_i - \prod_{i=1}^N b_i} \leq \sum_{i=1}^N \abs{a_i - b_i}
\end{equation}

Esta propiedad puede demostrarse por el principio de inducción.

i) $P(2)$
\begin{equation}
   \abs{a_1 a_2 - b_1 b_2 } \leq \abs{a_1 - b_1} + \abs{a_2 - b_2}
\end{equation}
Primero sumamos y restamos el término $a_2 b_1$ y luego sacamos factor común según:
\begin{equation*}
    \abs{a_1 a_2 - b_1 b_2} = \abs{a_1 a_2 - a_2 b_1 - b_1 b_2 + a_2 b_1} =  \abs{a_2 (a_1-b_1) + b_1 (a_2-b_2)} 
\end{equation*}
luego aplicamos la desigualdad triangular teniendo en cuenta además que $\abs{a_i}\leq 1$ y $\abs{b_i} \leq 1$
\begin{equation}
    \abs{a_1 a_2 - b_1 b_2} \leq \abs{a_2} \abs{a_1-b_1}+\abs{b_1}\abs{a_2-b_2} \leq \abs{a_1-b_1}+\abs{a_2-b_2}
\end{equation}

ii) $P(n) \Rightarrow P(n+1)$
\begin{equation}
    \abs{\prod_{i=1}^{N+1} a_i -\prod_{i=1}^{N+1} b_i  }  =  \abs{a_{N+1} \prod_{i=1}^{N} a_i - b_{N+1} \prod_{i=1}^{N} b_i } 
\end{equation}
Aplicando $P(2)$
\begin{equation}
\abs{\prod_{i=1}^{N+1} a_i -\prod_{i=1}^{N+1} b_i  }  \leq \abs{a_{N+1}-b_{N+1}} + \abs{\prod_{i=1}^{N} a_i - \prod_{i=1}^{N} b_i} 
\end{equation}
Considerando que $P(n)$ se cumple por hipótesis
\begin{equation}
\abs{\prod_{i=1}^{N+1} a_i -\prod_{i=1}^{N+1} b_i  }  \leq \abs{a_{N+1}-b_{N+1}} + \abs{\sum_{i=1}^{N} \abs{a_i - b_i}} =  \abs{\sum_{i=1}^{N+1} \abs{a_i - b_i}}
\end{equation}










\paragraph{Propiedad 2} El desarrollo en serie de Taylor, de $\sqrt{1-x}$ cuando $0 \leq x \leq 1$ es:
\begin{equation}
    \sqrt{1-x}= 1-\sum_{n=1}^{+\infty} f_n \cdot x^n \quad \text{con} \quad f_n = \frac{(2n)!}{(n!)^2 (2n-1) 4^n} \geq 0
    \label{eqn:taylorraiz}
\end{equation}

Esta propiedad se deduce de aplicar la fórmula del cálculo de coeficientes para el desarrollo de Serie de Taylor.

\begin{equation*}
    f(x) = f(x_0) + \sum_{n=1}^{+\infty} \frac{f^{(n)}(x_0)}{n!} \cdot (x-x_0)^n
\end{equation*}

Calculamos las derivadas n-ésimas para evaluarlas luego en $x=0^+$
\begin{align*}
    &f(x) = (1-x)^{1/2}\\
    &f'(x) = \frac{1}{2} (-1) (1-x)^{-1/2} = - \frac{1}{2}\\
    &f''(x) = \frac{1}{2} (-1) (-\frac{1}{2})(-1) (1-x)^{-3/2} = -\frac{1}{2} \frac{1}{2} (1-x)^{-3/2} \\
    &f'''(x) = -\frac{1}{2} \frac{1}{2} \frac{3}{2} (1-x)^{-5/2} \\
    &f^{(iv)}(x) = -\frac{1}{2} \frac{1}{2} \frac{3}{2} \frac{5}{2}  (1-x)^{-7/2} \\
    &f^{(n)}(x) = -\frac{1}{2} \frac{1}{2} \frac{3}{2} \frac{5}{2} ... \frac{2n-3}{2} (1-x)^{\frac{2n-1}{2}}
\end{align*}
completamos la sucesión presente en el numerador con los términos pares y hasta el término $2n$
\begin{align*}
    f^{(n)}(x) &= -\frac{1}{2} \frac{1}{2} \frac{3}{2} \frac{5}{2} ... \frac{2n-3}{2} (1-x)^{\frac{2n-1}{2}} = \\
      &= -\frac{1}{2} \frac{1}{2} \frac{2}{2} \frac{3}{2} \frac{4}{4} \frac{5}{2} \frac{6}{6} ... \frac{2n-3}{2} \frac{2n-2}{2n-2} \frac{2n-1}{2n-1} \frac{2n}{2n} (1-x)^{\frac{2n-1}{2}2} = \\
        &= -\frac{1}{2} (2n)!  \frac{1}{2} \frac{1}{2} \frac{1}{2} \frac{1}{4} \frac{1}{2} \frac{1}{6} ... \frac{1}{2} \frac{1}{2n-2} \frac{1}{2n-1} \frac{1}{2n} (1-x)^{\frac{2n-1}{2}} =\\
      &= -\frac{1}{2} (2n)!   \frac{1}{4}   \frac{1}{4\cdot2} \frac{1}{4\cdot3} ... \frac{1}{4\cdot(n-1)} \frac{1}{2n-1} \frac{1}{2n} (1-x)^{\frac{2n-1}{2}} =\\        
            &= - (2n)!    \frac{1}{4}   \frac{1}{4\cdot2} \frac{1}{4\cdot3} ... \frac{1}{4\cdot(n-1)} \frac{1}{4\cdot n} \frac{1}{(2n-1)} (1-x)^{\frac{2n-1}{2}} = - \frac{(2n)!}{n! (2n-1) 4^n}
\end{align*}

Finalmente obtenemos el desarrollo en serie indicado evaluando $f^{(n)}$ en $x=0^+$


\paragraph{Propiedad 3} Si $u$ y $v$ son tales que $0 \leq u \leq 1 \ , \ 0 \leq v \leq 1$ y dados $A,B \in \reals$ 

\begin{equation}
    \abs{A \sqrt{1-u} - B \sqrt{1-v} } \leq \abs{A-B} + \sum_{n=1}^{\infty} f_n \left(\abs{A} u^n + \abs{B} v^n\right)
    \label{eqn:desigtriangconproductoria2}
\end{equation}

Como $A$ y $B$ pueden ser mayores que 1 (en módulo), no se puede aplicar directamente la propiedad 1, con lo cual primero reemplazamos las raíces con sus desarrollos en serie y luego se distribuye.

\begin{equation}
    \abs{A \sqrt{1-u} - B \sqrt{1-v} } \leq \abs{A-B} + \sum_{n=1}^{\infty} f_n \left(\abs{A} u^n + \abs{B} v^n\right)
    \label{eqn:desigtriangconproductoria3}
\end{equation}


\bibliographystyle{plain}
\bibliography{biblio}

\end{document}