\section{Introducción}
En este trabajo se caracterizará estadísticamente un tren de disparos (potencial de acción) para una red neuronal del tipo Integración y Disparo con Perdida de memoria (IDP). Una red neuronal de este tipo está constituida por neuronas individuales e independientes interconectadas entre si, entrada-salida, con la capacidad de generar potenciales de acción (PA) en respuesta a los estímulos proveniente del resto de las neuronas de la red. El potencial de acción es el estado de activación de una neurona. Una neurona se modela matemáticamente en tiempo discreto como un sumador de los estados de la entrada, donde un valor fijo denominado umbral determina cuándo una neurona desarrolla un PA. Las neuronas están interconectadas entre sí mediante pesos sinápticos que regulan cuánto afecta el PA de una neurona sobre el estado de otra neurona. Las interacciones entre las neuronas describen la dinámica de la red neuronal y de esta forma si las interacciones no presentan ruido, el estado futuro de la red podría ser perfectamente determinado en base a los estados anteriores. Biológicamente, las secuencias de potenciales de acción representa información almacenada en la red, una conexión estímulo-respuesta que en algún tiempo pasado fue aprendido o heredado. Sin embargo, los registros obtenidos bajo las mismas condiciones experimentales no describen secuencias exactas de disparos, demandando con ello la búsqueda de regularidades estadísticas para la caracterización de una red neuronal real. Un forma de acercarse al comportamiento biológico mediante el modelo de IDP es introduciendo ruido en el modelo. 

Uno de los principales obstáculos para caracterizar una red neuronal bajo una secuencia de estímulos es la necesidad de conocer el estado inicial. Considerando que una red neuronal no comienza en el momento del registro sino al comienzo del desarrollo de la vida, es estado inicial tendrá una distribución probabilidad desconocida.

En este trabajo se presentará una caracterización completa del tren de disparos utilizando un modelo de IDP con ruido gaussiano en tiempo discreto y un estímulo externo constante. El trabajo se organiza de la siguiente forma: Probabilidad de que la neurona no dispare a tiempo $t$ dado el pasado, Probabilidad de transición no Markoviana, Existencia de una medida única e invariante, Aproximación Markoviana con memoria a $R$ estados y aplicaciones en neurociencia. 