\section{Potencial Regular}


Una función $\psi$ es un potencial regular si es una función continua y cumple con \cite{keller_equilibrium_1998}

\begin{align*}
\left. \begin{array}{l}
     \psi : \Omega \to \reals  \\
     \psi \text{ es } C^0 \\ 
     \displaystyle \sum_{k\geq0} var_k(\psi) < \infty
\end{array} \right\}
\Longrightarrow \psi \text{ es un potencial complejo}
\end{align*}

\subsection{Entropía}

Dada una medida $\mu \in M_T(\Omega)$ donde $M_T(\Omega)$  es el set de medidas de probabilidad invariantes ante la transformación T, se define la entropía de la medida como:

\begin{equation}
    h(\mu) = \limsup{\frac{1}{n+1} \sum_{[\omega]_0^n} \mu([\omega]_0^n) \log(\mu([w]_0^n))}
    \label{eqn:entropia}
\end{equation}

donde $[w]_0^n$ son todos los cilindros de longitud $n+1$

\paragraph{Definición 5} Se denomina $\mu_\psi$, un estado de equilibro del potencial $\psi$, a la medida invariante respecto a $T$ sobre $\Omega$ tal que:

\begin{equation}
    P(\psi) = h(\mu_\psi) + \mu_\psi (\psi) = \sup_{}\mu \in M_T(\Omega) \left\{ h(\mu)+\mu(\psi)\right\}
    \label{eqn:presiontopo}
\end{equation}

La presión topológica es cero cuando el potencial $\psi$ está normalizado. Es el caso de un potencial $\psi$ como el que utilizaremos ya que será el logaritmo de una probabilidad condicional\cite{keller_equilibrium_1998}.

Si $\psi$ es un potencial regular, entonces los estados de equilibrios de $\psi$ son las $g$-medidas asociadas a una $g$-función continua. En nuestro caso, donde sabemos que la $g$-medida es única, la función $g_0$ se relaciona con el potencial según

\begin{equation}
    \psi(\omega) = \log[g_0(\omega)]
\end{equation}


\paragraph{Teorema 2} Para cualquier set de parámetros ($W_{i,j}$, $\gamma$, $\umbral$,
$\sigma_B$) el sistema tiene una $g$-medida única, y es un estado de equilibro
para el potencial $\psi=\log[g_0(\omega)]$



\begin{align*}
  \psi(\omega) & = \psi(\omega_{-\infty}^0)=\log[g_0(\omega)]= \\
  & = \log\left[\prod_{i=1}^N\left[\omega_i(0) \ \Pi\left(\frac{\theta-C_i(\uom)}{\sigma_i(\uom)}\right)+(1-\omega_i(0))\left(1-\Pi\left(\frac{\theta-C_i(\uom)}{\sigma_i(\uom)}\right)\right)\right]\right] = \\
  & = \sum_{i=1}^N \log\left[\omega_i(0) \ \Pi\left(\frac{\theta-C_i(\uom)}{\sigma_i(\uom)}\right)+(1-\omega_i(0))\left(1-\Pi\left(\frac{\theta-C_i(\uom)}{\sigma_i(\uom)}\right)\right)\right]
\end{align*}

como $\omega(0)$ es 0 o 1,

\begin{equation}
  \psi(\omega) = 
  \sum_{i=1}^N \left[\omega_i(0) \log\left[ \Pi\left(\frac{\theta-C_i(\uom)}{\sigma_i(\uom)}\right)\right]+(1-\omega_i(0))\log\left[1-\Pi\left(\frac{\theta-C_i(\uom)}{\sigma_i(\uom)}\right)\right]\right] =
  \label{potencailnu}
\end{equation}


\paragraph{Proposición 9} El potencial de membrana $V(t)$ es estacionario con densidad de probabilidad producto

\begin{equation*}
  \rho_V(v) = \prod_{i=1}^N \rho_{V_i}(v_i) 
\end{equation*}

donde

\begin{equation*}
  \rho_{V_i}(v_i) = \int_{\uOm} \frac{1}{\sqrt{2 \pi \sigma_i(\uom)}} e^{-\frac{1}{2}\left(\frac{v-C_i(\uom)}{\sigma_i(\uom)}\right)^2} \mu_\psi(d\uom)
\end{equation*}

donde su esperanza es $\mu_\psi[C_i(\uom)]$ y la varianza es $\mu_\psi[\sigma_i^2(\uom)]$


\subsection{Tasa de disparos}

\begin{equation*}
  r_i(\omega) = P(\omega_i(0)=1 | \uom) = \Pif
\end{equation*}


\begin{equation*}
  r_i = \mu_\psi(\omega_i(0)=1)
\end{equation*}

\begin{equation*}
  r_i = \mu_\psi (r_i(\omega))=\Pif
\end{equation*}



\subsection{Entropía}

Como el potencial $\psi$ está normalizado $P(\psi)=0=h(\mu_\psi)+\mu_\psi(\psi)$

\begin{align}
   h(\mu_\psi) & = - \sum_{i=1}^N \mu_\psi \left[ \omega_i(0) \log\left(\Pif\right) + (1-\omega_i(0)) \log\left( 1-\Pif\right) \right]  \nonumber \\
  & = - \sum_{i=1}^N \left[\mu_\psi(\omega_i(0)) \ \mu_\psi\left( \log r_i(\omega) \right)+ \mu_\psi(1-\omega_i(0)) \ \mu_\psi \left(\log\left( 1-r_i(\omega) \right) \right) \right] \nonumber\\
  & = - \sum_{i=1}^N \left[ r_i \ \mu_\psi\left( \log r_i(\omega) \right)+ (1-r_i) \ \mu_\psi \left(\log ( 1-r_i(\omega) ) \right) \right] \label{eqn:entro1}
\end{align}
  
\paragraph{Proposición 10} La entropía del estado de equilibrio es positiva para cualquier conjunto de parámetros del sistema.

Debido a que $\psi(\omega) < 0 \  \forall \  \omega \in \Omega$ entonces $\mu_\psi(\psi) < 0$. A partir de la ecuación \eqref{eqn:entro1} vemos que la entropía es siempre positiva.


\subsection{Divergencia de Kullback-Leibler}

Dadas dos medidas T-invariantes $(\mu,\nu)$ la diveregencia de Kullback-Leibler se define como

\begin{equation}
    d_{KL}(\mu,\nu) = \lim_{n\to\infty} \frac{1}{n+1} \sum_{[w]_0^n} \mu([w]_0^n) \log \left( \frac{\mu([w]_0^n)}{\nu([w]_0^n)} \right)
\end{equation}


Para una medida $\mu$ ergódica y $\mu_\psi$ un estado de gibbs del potencial $\psi$

\begin{equation}
    d_{KL}(\mu,\mu_\psi) = p(\psi)-\mu(\psi)-h(\psi) = h(\mu_\psi)+\mu_\psi(\psi)-(h(\mu)-\mu(\psi)) = (h(\mu_\psi) - h(\mu)) + (\mu_\psi(\psi) - \mu_(\psi))
\end{equation}

\subsection{Estados de Gibbs}

Los estados de equilibrio de un potencial regular son las medidas de Gibbs o estados de Gibbs \cite{keller_equilibrium_1998}. Además, existe una constante $C_\psi >0$ tal que

\begin{equation}
    0 \leq e^{-C_\psi} \leq \frac{\mu_\psi([\omega]^n_0)}{e^{-(n+1)p(\psi)} e^{\sum^n_{k=0}\psi(T^k\omega)}} \leq e^{C_\psi}
\end{equation}

Como el potencial $\psi$ está normalizado, $p(\psi)=0$.
Esto implica que la medida toma la forma 

\begin{equation}
    \mu_\psi([\omega]^n_0) ==\frac{e^{\sum^n_{k=0}\psi(T^k\omega)}}{Z_\psi^{n+1}(\omega)}
\end{equation}

donde $Z_\psi^{n+1}(\omega)$ es análoga a la función de partición aunque en este caso depende de $\omega$

\nocite{keane_strongly_1972,ledrappier_principe_1974,johansson_square_2003,cessac_discrete_2010}
